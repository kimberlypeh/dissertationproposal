\documentclass [11pt]{article}

\title{Post-Conflict Democracy Index (PCDI): V-Dem's Electoral Democracy Index Reconceptualized} 
\author{Kimberly Peh}
\date{\today}

\usepackage{hanging,verbatim,geometry,rotating,graphics,epigraph,afterpage,url,pdfpages,pifont}
\usepackage{etex,tikz,xcolor,verbatim,geometry,afterpage,float}
\usepackage{pgfplots}
\usepgflibrary{arrows.meta} % need this for arrow tips
\pgfplotsset{width=10cm,compat=1.14}
\usetikzlibrary{datavisualization}
\usepgfplotslibrary{statistics} 
\usepackage{Times}

\geometry{height=8in, width=5in}

\usepackage[utf8]{inputenc}

\setlength{\epigraphwidth}{.8\textwidth} \setlength{\epigraphrule}{0pt}

\urlstyle{same}

\lefthyphenmin=2
\righthyphenmin=3

\brokenpenalty=10000 % No broken words across columns/pages
\widowpenalty=10000 % No widows at bottom of page
\clubpenalty=10000 % No orphans at top of page

\begin{document}
\maketitle

\section*{Introduction} 

\epigraph{Democracy is an ingredient for both sustainable development and lasting peace.}{\emph{Boutros Boutros-Ghali}}

Since the end of the Cold War, democracy promotion became a norm in the peacebuilding circle. The United Nations (UN) regards democracy as a contributor to lasting peace and hence, facilitates democratization on the one hand and incorporates democratic tools into peacekeeping operations on the other. The United States Agency for International Development (USAID) claims democracy as a ``necessary'' condition for conflict prevention and thus contingents aid on states' commitment towards democracy. By the early 1990s, two democratic tools -- the holding of elections (often within an arbitrary period of two years) and the writing of a contitution -- became the bread and butter of conflict resolution efforts and they continue to form the foundation of peacebuilding today (Ottaway 2003). Even amid debates over the influence of democracy promotion on the durability of post-conflict peace, scholars almost never question \emph{whether} democracy should be pursued. Rather, their emphases on sequencing and investigations on the conditions that are advantageous to the pursuit of democracy reveal that they, too, are supportive of democracy as they caution against undermining the other goal, that is, sustaining peace (Joshi, Melander and Quinn 2017; Brancati and Snyder 2012; Flores and Nooruddin 2012; Joshi 2010; Paris 2004).

This view of democracy as a force for conflict resolution and mediation rests on several empirical and theoretical bases. Empirically, democracies are associated with the absence of repression and large-scale human rights violation. Internationally, democracies are virtually never at war with each other. Thus, according to some scholars, democracies are critical for achieving greater domestic and international peace (Davenport; Bruce and Russett). Theoretically, democracy is expected to dampen conflicts because it provides a peaceful and transparent means of succession (i.e. elections); allows room for diverse opinions; and resolves conflicts through accommodation (Mross 2019; Joshi 2010). In post-civil war countries, where democracy is most widely promoted since the 1990s, the conduct of elections and the writing of constitutions are furthermore perceived to be desirable because they legitimize governments; institutionalize means for further democratization and conflict resolution; encourage respect for human rights; signal commitment towards peace; and attract foreign aid (Brancati and Snyder 2012; Joshi 2010).

Scholarly works that bring together the conflict and the democracy literatures are therefore hardly a scarcity. Yet, an area that remains lacking is a discussion of the measurement problems that underlie these substantive debates. The majority of the works in the conflict literature uses data from the Polity project when measuring the concept of democracy. However, the deficiencies of the project's democracy index, Polity2, are well recognized in the democracy literature. More recent analyses of the existing democracy indices suggest that a better indicator is the Electoral Democracy Index (EDI), which is derived from the Varieties of Democracy (V-Dem) project. Even then, changes need to be made to the index for it to meet the purpose of promoting democratization, and that is, to induce non-violence, in post-conflict settings. This chapter takes on this task of re-inventing V-Dem's EDI, which it renames as the Post-Conflict Democracy Index (PCDI). In PCDI, the dimensions of violence and non-violence are reconstituted as necessary conditions of democracy to allow users to evaluate whether post-conflict states are moving towards the goal of democracy \emph{and} becoming more non-violent. The latter assessment becomes possible with PCDI because its emphasis on violence and non-violence results in progression towards democracy only if states are engaging is less violence. Such violence refers specifically to behaviors that manifest in the threat or the actual use of force which diminishes the meaningfulness of elections. Examples include assaults on journalists to limit media freedom and state or non-state attacks on voters to sway vote choice. Violence that compromises the quality of elections are particularly detrimental to democratic progress because free and fair elections are a fundamental component of democracies. If post-conflict states cannot at minimum maintain a non-violent electoral arena, any democratic gains may be superficial and indeed, hinder peace, as advocates of the institutionalization-before-democratization approach suggest.

The next section moves the discussion forward by outlining the issues inherent to Polity's and V-Dem's democracy indices. Then, the third section follows by putting forward this chapter's proposed index, the PCDI, with a detailed explanation of why it is more suitable for measuring democratic progress in post-conflict settings and how it can be constituted by improving upon V-Dem's Electoral Democracy Index. The fourth section describes how PCDI would look like across a scale, and closes this chapter with an overview of how the index would be used in the subsequent chapters.

\section*{Measurement Issues: Polity and V-Dem}








































\section*{References} 

\begin{hangparas}{1cm}{1}

\end{hangparas}




\end{document}


















