\documentclass [11pt]{article}

\title{Post-Conflict Democracy Index (PCDI): VDem's Electoral Democracy Index Reconceptualized} 
\author{Kimberly Peh}
\date{\today}

\usepackage{hanging,verbatim,geometry,rotating,graphics,epigraph,afterpage,url,pdfpages,pifont}
\usepackage{etex,tikz,xcolor,verbatim,geometry,afterpage,float}
\usepackage{pgfplots}
\usepgflibrary{arrows.meta} % need this for arrow tips
\pgfplotsset{width=10cm,compat=1.14}
\usetikzlibrary{datavisualization}
\usepgfplotslibrary{statistics} 
\usepackage{Times}

\geometry{height=8in, width=5in}

\usepackage[utf8]{inputenc}

\setlength{\epigraphwidth}{.8\textwidth} \setlength{\epigraphrule}{0pt}

\urlstyle{same}

\lefthyphenmin=2
\righthyphenmin=3

\brokenpenalty=10000 % No broken words across columns/pages
\widowpenalty=10000 % No widows at bottom of page
\clubpenalty=10000 % No orphans at top of page

\begin{document}
\maketitle

\section*{Introduction} 

\epigraph{Democracy is an ingredient for both sustainable development and lasting peace.}{\emph{Boutros Boutros-Ghali}}

Since the end of the Cold War, democracy promotion became a norm in the peacebuilding circle. Regarding democracy as an ingredient for lasting peace (Boutros-Ghali, 1996), the United Nations (UN) has, since the 1990s, facilitated democratization on the one hand while incorporating democratic tools into peacekeeping operations on the other. Reflecting the United States' ideology, the United States Agency for International Development (USAID) claims democracy as a ``necessary'' condition for conflict prevention and contingents aid on states' commitment towards democracy. By the early 1990s, two democratic tools -- the holding of elections (often within an arbitrary period of two years) and the writing of contitutions -- became the bread and butter of conflict resolution efforts, and they continue to form the foundation of peacebuilding today (Ottaway 2003). Even amid debates over the influence of democracy promotion on the durability of post-conflict peace, scholars almost never question \emph{whether} democracy should be pursued. Rather, their emphases on sequencing and investigations of the conditions that are advantageous to democracy reveal that they, too, are supportive of democracy even as they caution against undermining the other goal, that is, sustaining peace (Joshi, Melander and Quinn 2017; Brancati and Snyder 2012; Flores and Nooruddin 2012; Joshi 2010; Paris 2004).

This view of democracy as a force for conflict resolution and mediation rests on several empirical and theoretical bases. Empirically, democracies are associated with the absence of repression and large-scale human rights violation. Internationally, democracies are virtually never at war with each other. Thus, according to some scholars, democracies are critical for achieving greater domestic and international peace (Davenport; Bruce and Russett). Theoretically, democracy is expected to dampen conflicts because it provides a peaceful and transparent means of succession (i.e. elections); allows room for diverse opinions; and resolves conflicts through accommodation (Mross 2019; Joshi 2010). In post-civil war countries, where democracy is most widely promoted since the 1990s, the conduct of elections and the writing of constitutions are furthermore perceived to be desirable because they legitimize governments; institutionalize means for further democratization and conflict resolution; encourage respect for human rights; signal commitment towards peace; and attract foreign aid (Brancati and Snyder 2012; Joshi 2010).

Scholarly works that bring together the conflict and the democracy literatures are therefore hardly a scarcity. Yet, an area that remains lacking is a discussion of the measurement problems that underlie these substantive debates. The majority of the works in the conflict literature uses data from the Polity project when measuring the concept of democracy. However, the deficiencies of the project's democracy index, Polity, are well recognized in the democracy literature. More recent analyses of the existing democracy indices suggest that a better indicator is the Electoral Democracy Index (EDI), which is generated by the Varieties of Democracy (V-Dem) project. Even then, changes need to be made to the index for it to better capture the fundamental purpose of democracy promotion, and that is, to induce non-violence, in post-conflict settings. This chapter takes on the task of re-inventing V-Dem's EDI, which it renames as the Post-Conflict Democracy Index (PCDI). In PCDI, the dimensions of violence and non-violence are reconstituted as necessary conditions of democracy to allow users to evaluate whether post-conflict states are moving towards the goal of democracy \emph{and} becoming more non-violent. The latter assessment is made possible with PCDI because its emphasis on violence and non-violence allows countries to attain a higher level of democracy only if states are engaging in less violence. Such violence refers specifically to behaviors that manifest in the threat or the actual use of force which diminishes the meaningfulness of elections. Examples include assaults on journalists to limit media freedom and state or non-state attacks on voters to sway vote choice. Violence that compromises the quality of elections are particularly detrimental to democratic progress because free and fair elections are a fundamental component of democracies. If post-conflict states cannot at minimum maintain a non-violent electoral arena, any democratic gains may be superficial and may indeed, hinder peace, as advocates of the institutionalization-before-democratization approach suggest.

The next section moves the discussion forward by outlining the issues inherent to Polity's and V-Dem's democracy indices. Then, the third section follows by putting forward this chapter's proposed index, the PCDI, with a detailed explanation of why it is more suitable for measuring democratic progress in post-conflict settings and how it can be constituted by improving upon V-Dem's Electoral Democracy Index. The fourth section describes how PCDI would look like across a theoretical scale, and closes this chapter with an overview of how the index would be used in the subsequent chapters.

%not just a normative pursuit, but a causal mechanism expectation (desire to leverage the non-violence dimension of democracy)

\section*{Measurement Issues: Polity and Varieties of Democracy}

 \subsection*{Polity}

Polity is the most widely used measure of democracy in the conflict literature. It contains a comprehensive set of data on 195 countries from 1800 to 2016, and grounds its conceptual basis in Harry Eckstein and Ted Robert Gurr's (1975) \emph{Patterns of Authority}, which characterizes democracies and autocracies as ``distinct patterns of authority'' (Marshall, Gurr and Jaggers 2017, 17). Despite its original intent, scholars often derive an overall Polty score by subtracting the computed autocracy (AUTOC) scores from the democracy (DEMOC) scores. This move transforms the Polity score into an interval variable (ranging from -10 to 10), which gives rise to issues that are compounded by the measure's inherent conceptualization, measurement and aggregation problems (for a critique of Polity and other democracy indices, see Boese 2019; Treir and Jackman 2008; Munck and Verkuilen 2002). In the list of weaknesses, three are critical to limiting the suitability of Polity as a measure of post-conflict progress.

First, Polity has, as a part of its scoring criteria, aspects that capture civil conflict. For instance, in the guidelines for coding the sub-dimensions of a regime's regulation (PARREG) and competitiveness (PARCOMP) of participation, Marshall et al (2017, 75) states that ``[a] polity is coded [as an instance of factional competition] if democratic elections are held in an environment of persistent and widespread civil unrest (rebellion, revolution, and/or ethnic conflict).''' Likewise, when coding whether polities are undergoing ``persistent overt coercion'', Marshall et al (2017, 77) writes that ``[a] polity is coded here if elections are deemed to be “unfair” because ... democratic elections are held in an environment of a persistent, yet largely ineffective or waning, civil violence or ethnic conflict.'' Unlike violence which takes place during an electoral cycle, civil conflict is neither a constitutive dimension of democracy nor autocracy because its ocurrence does not reveal the nature of a state's regime by definition. Its inclusion as a coding criterion incorporates irrelevant dimensions into the measurement, and more importantly, hinders an empirical study of the relationship between conflict and regime. Stated in statistical terms, what this coding rule does is introduce the same empirical evidence into the left and the right hand sides of a model. To some extent, since these coding rules confine conflict states to the middle of the Polity scores, the consistent correlation between anocracies and violence, or the so-called murder in the middle, might therefore be an artefact of such coding decisions (Boese 2019). In one study, the inverse U-shaped relationship between conflict and regime in fact disappears once Polity's aggregation method is changed from summation to the weighted average and when measurement error is incorporated into the model (Trier and Jackman 2008).

Second, because of Polity's weighting and aggregation decisions, substantive interpretation can be hampered and a wide variation of countries can be grouped within one regime score (Boese 2019; Trier and Jackman 2008; Munck and Verkuilen 2002). Overall, three dimensions, namely, executive recruitment; executive constraints; and political competition, undergird each regime type in the Polity project. Each of these dimensions is, in turn, constituted by a different number of sub-dimensions and their nominal categories. To compute the final democracy and autocracy scores, the Polity project sums up the sub-dimensions along with the weights that are assigned to the majority of these nominal categories. Figure 1 summarizes the subdimensions, the categories and the scale weights. These weights are an issue because they create problems for those who seek to glean substantive interpretations from a given Polity score. When both the sub-dimensions and the weights are simply aggregated through summation, every point carried by these weights become valued at the same level as the sub-dimensions. Yet, because the coding manual neither explains the rationale for the weight assignment nor provides a justification for the assigned values, there is no clear way to make sense of the substantive meaning of a final score. Moreover, because these weights increase the number of configurations that can make up the final score, each given level along the Polity index thus comprise a greater range of polities. The result is a coarsening of the information which each Polity score conveys. Polity is therefore unsuitable for capturing post-conflict democratic progress since fine differences matter in differentiating among complex political environments and in informing policy makers on important issues of resource allocation.
% insert Figure 1: Boese's Table 4, p. 103

Third and finally, Polity is an inadequate measure of post-conflict progress because it largely omits an important dimension of democracy, that is, participation (Boese 2019; Munck and Verkuilen 2002). According to the coding manual, participation is operationalized in two ways. The first is ``regulation of participation'', which identifies the extent to which ``there are binding rules on when, whether, and how political preferences are expressed.'' Countries are assigned to different nominal categories based on the extent of factionalism and the degree to which groups are excluded from political involvement (Marshall et al. 2017, 26). The second is ``competitiveness of participation'', which specifies the extent to which diverse opinions are tolerated. Countries are classified under less competitive categories if states and political parties act on repression, exclusion or parochialism (Marshall et al. 2017, 26-7). These coding rules make clear that what is really coded in Polity is exclusion and tolerance, and not participation, which is taken by democracy scholars to mean the extent of inclusion or the degree to which citizens may hold politicians accountable (Held 2006; Dahl 1998). The ability to measure inclusion rather than exclusion is particularly crucial for measuring post-conflict democratic progress. Conflicts oftentimes break out because of exclusive politics; hence, the design of such institutions (e.g. political systems, electoral designs, and constitutions) that influence the extent of inclusivity is a huge matter of contention in post-conflict politics. Without measures capable of picking up on such advancement, Polity leaves out an important assessor of post-conflict democratic peace.
% Other issues with polity: Nominal; Equal weighting (problem of equivalence); Redundancy due to lack of clarity in the levels of abstraction/dimensions; Substantive interpretations of the 0 category

\subsection*{Varieties of Democracy (VDem) and the Problem with the Latent Variable Approach}

VDem's Electoral Democracy Index (EDI) measures the extent to which a country resembles an ideal electoral democracy. By electoral democracy, VDem is referring to Robert Dahl's (1998) conceptualization of polyarchy, which has, as necessary conditions, six key political institutions: (1) elected officials; (2) free, fair, and frequent elections; (3) freedom of expression; (4) alternative sources of information; (5) associational autonomy; and (6) inclusive citizenship. EDI collapses the third and the fourth dimensions into one, and thus, is constituted by a total of five attributes. Out of the existing democracy indices, the EDI receives a particularly high praise because its measurement is built upon a clear definition; its aggregation rule is clearly outlined and justified; inter-coder reliability is well taken into account; and its use of Bayesian updating allows the index to capture new information that may vary over time (Boese 2019). For the purpose of this project, these improvements are particularly crucial because they resolve the above issues that constrain Polity's suitability as a measure of post-conflict democracy. Unnecessary variables like civil conflict are left outside of EDI's measurement, and both dimensions -- competition and participation -- are clearly represented. Moreover, because VDem's aggregation rules are well-specified, every unit change along the interval is substantively meaningful and interpretation becomes less of an issue.

However, for all its advantages, an issue remains with the use of the latent variable approach in the construction of the EDI. The latent variable approach is oftentimes chosen for its ability to simulate uncertainty and weigh indicators using a data-driven process (Smith and Spaniel 2020; Treir and Jackman 2008). Accounting for uncertainty, and hence, measurement error, is particularly important when the latent variable is to be used as an independent variable because statistical models rarely consider errors on the left hand side of the model. A consequence of such omission is the potential biasing of parameter estimates, which in turn, leads scholars to arrive at wrong conclusions (Trier and Jackman 2008). The simulation of uncertainty, furthermore, allows scholars to accurately quantify the level of confidence which prevents conclusions from being over-stated. The data-driven approach to estimating indicator weights can also be advantageous because it replaces the need to assign arbitrary weights and instead lets the data reveal relationships that exist within itself. These benefits notwithstanding, the latent variable approach is inappropriate when the variable at hand is one with non-compensatory dimensions (Wuttke, Schimpf and Schoen 2020).

A typical approach to aggregating the weighted indicators is by averaging them. The final outcome is thus a weighted average of all the indicators, which in the EDI, occurs at the level of aggregation from the indicator level to the conceptual attributes. The sole exception is the dimension of inclusive citizenship, which includes only the extent of suffrage. An implication of applying averages is that higher scores on one indicator can be used to compensate for lower scores on the others because it is the mean value that determines the final score (Goertz 2020; Wuttke, Schimpf and Schoen 2020). This effect of substitutability among indicators leads to concept-measure inconsistency when dimensions are conceptualized as necessary conditions because indicators are not supposed to be substitutable. Under such a condition, a unit drops out as an instance of the overall concept as soon as it fails to fulfil at least one of the necessary dimensions. When it comes to analyzing post-conflict democratic progress, two such constitutive and non-substitutable dimensions are violence and non-violence. 

The pursuit of democracy in post-conflict settings may be described as a normative goal (Flores and Nooruddin 2012). However, the same goal may be sought too for its relatively non-violent tendencies when compared with other regime types. The language used within the peacebuilding circle when describing the relationship between democracy and peace certainly evidences the expectation of such a theoretical relationship that connects the seemingly separate goals. In the words of Boutros Boutros-Ghali (1996, 8), a former Secretary-General of the UN, ``Democracy within States ... fosters the evolution of the social contract upon which lasting peace can be built.'' Furthermore, some scholars maintain that because democracy and its tools `'provide insitutional mechanisms and legitimacy,'' it is a necessary condition ``without which peace building in a post-war state may not materialize (Joshi 2010, 829).'' Therefore, for a measure of post-conflict democratic progress to work effectively as an assessment tool and as a potential independent variable, both dimensions of violence and non-violence need to be clearly specified as non-compensatory attributes. In this way, intransigent states and actors may not be classified as instances of progress, and those which succeed in pursuing non-violence may be identified as cases to further explore the relationship between democracy and peace.
%fully develop the issue with the latent variable approach

\section*{Constructing the Post-Conflict Democracy Index (PCDI)}

%Concept; Measure; Aggregation; Scale







































\section*{References} 

\begin{hangparas}{1cm}{1}

\end{hangparas}




\end{document}


















