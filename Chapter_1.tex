\documentclass [11pt]{article}

\title{Post-Conflict Democracy Index (PCDI): VDem's Electoral Democracy Index Reconstructed} 
\author{Kimberly Peh}
\date{\today}

\usepackage{hanging,verbatim,geometry,rotating,graphics,epigraph,afterpage,url,pdfpages,pifont}
\usepackage{etex,tikz,xcolor,verbatim,geometry,afterpage,float}
\usepackage{pgfplots}
\usepgflibrary{arrows.meta} % need this for arrow tips
\pgfplotsset{width=10cm,compat=1.14}
\usetikzlibrary{datavisualization}
\usepgfplotslibrary{statistics} 
\usepackage{Times}

\geometry{height=8in, width=5in}

\usepackage[utf8]{inputenc}

\setlength{\epigraphwidth}{.8\textwidth} \setlength{\epigraphrule}{0pt}

\urlstyle{same}

\lefthyphenmin=2
\righthyphenmin=3

\brokenpenalty=10000 % No broken words across columns/pages
\widowpenalty=10000 % No widows at bottom of page
\clubpenalty=10000 % No orphans at top of page

\begin{document}
\maketitle

\section*{Introduction} 

\epigraph{Democracy is an ingredient for both sustainable development and lasting peace.}{\emph{Boutros Boutros-Ghali}}
\epigraph{If only because democratic governments prevent abusive autocracies from ruling, they meet [the] requirement [of doing no harm] better than nondemocratic governments.}{\emph{Robert A. Dahl}}

Since the end of the Cold War, democracy promotion became a norm in the peacebuilding circle. Regarding democracy as an ingredient for lasting peace (Boutros-Ghali, 1996), the United Nations (UN) has, since the 1990s, facilitated democratization on the one hand while incorporating democratic tools into peacekeeping operations on the other. Reflecting the United States' ideology, the United States Agency for International Development (USAID) claims democracy as a ``necessary'' condition for conflict prevention and contingents aid on states' commitment towards democracy. By the early 1990s, two democratic tools -- the holding of elections (often within an arbitrary period of two years) and the writing of contitutions -- became the bread and butter of conflict resolution efforts, and they continue to form the foundation of peacebuilding today (Ottaway 2003). Even amid debates over the influence of democracy promotion on the durability of post-conflict peace, scholars almost never question \emph{whether} democracy should be pursued. Rather, their emphases on sequencing and investigations of the conditions that are advantageous to democracy reveal that they, too, are supportive of democracy even as they caution against undermining the other goal, that is, sustaining peace (Joshi, Melander and Quinn 2017; Brancati and Snyder 2012; Flores and Nooruddin 2012; Joshi 2010; Paris 2004).

This view of democracy as a force for conflict resolution and mediation rests on several empirical and theoretical bases. Empirically, democracies are associated with the absence of repression and large-scale human rights violation. Internationally, democracies are virtually never at war with each other. Thus, according to some scholars, democracies are critical for achieving greater domestic and international peace (Davenport; Bruce and Russett). Theoretically, democracy is expected to dampen conflicts because it provides a peaceful and transparent means of succession (i.e. elections); allows room for diverse opinions; and resolves conflicts through accommodation (Mross 2019; Joshi 2010). In post-civil war countries, where democracy is most widely promoted since the 1990s, the conduct of elections and the writing of constitutions are furthermore perceived to be desirable because they legitimize governments; institutionalize means for further democratization and conflict resolution; encourage respect for human rights; signal commitment towards peace; and attract foreign aid (Brancati and Snyder 2012; Joshi 2010).

Scholarly works that bring together the conflict and the democracy literatures are hardly a scarcity. Yet, an area that remains lacking is a discussion of the measurement problems that underlie these substantive debates. The majority of the works in the conflict literature uses data from the Polity project when measuring the concept of democracy. However, the deficiencies of the project's democracy index, Polity, are well recognized in the democracy literature. More recent analyses of the existing democracy indices suggest that a better indicator is the Electoral Democracy Index (EDI), which is generated by the Varieties of Democracy (V-Dem) project. Even then, changes need to be made to the index for it to better capture the fundamental purpose of democracy promotion, and that is, to induce non-violence, in post-conflict settings. This chapter takes on the task of re-inventing V-Dem's EDI, which it renames as the Post-Conflict Democracy Index (PCDI). In PCDI, the dimensions of violence and non-violence are reconstituted as necessary conditions of democracy to allow users to evaluate whether post-conflict states are moving towards the goal of democracy \emph{and} becoming more non-violent. The latter assessment is made possible with PCDI because its emphasis on non-violence allows countries to attain a higher level of democracy only if states are engaging in less violence. Such violence refers specifically to behaviors that manifest in the threat or the actual use of force which diminishes the meaningfulness of elections. Examples include assaults on journalists to limit media freedom and state or non-state attacks on voters to sway vote choice. Violence that compromises the quality of elections are particularly detrimental to democratic progress (i.e. advancement towards democracy) because free and fair elections are a fundamental component of democracies. If post-conflict states cannot at minimum maintain a non-violent electoral arena, any democratic gains may be superficial and may indeed, hinder peace, as advocates of the institutionalization-before-democratization approach suggest.

The next section moves the discussion forward by outlining the issues inherent to Polity's and VDem's democracy indices. Then, the third section follows by putting forward this chapter's proposed index, the PCDI, and explains how the index can be constituted by improving upon VDem's Electoral Democracy Index. The fourth section closes with some points on what more needs to be done to expand on and complete the chapter.

%not just a normative pursuit, but a causal mechanism expectation (desire to leverage the non-violence dimension of democracy)

\section*{Measurement Issues: Polity and Varieties of Democracy}

 \subsection*{Polity}

Polity is the most widely used measure of democracy in the conflict literature. It contains a comprehensive set of data on 195 countries from 1800 to 2016, and grounds its conceptual basis in Harry Eckstein and Ted Robert Gurr's (1975) \emph{Patterns of Authority}, which characterizes democracies and autocracies as ``distinct patterns of authority'' (Marshall, Gurr and Jaggers 2017, 17). Despite its original intent, scholars often derive an overall Polty score by subtracting the computed autocracy (AUTOC) scores from the democracy (DEMOC) scores. This move transforms the Polity score into an interval variable (ranging from -10 to 10), which gives rise to issues that are compounded by the measure's inherent conceptualization, measurement and aggregation problems (for a critique of Polity and other democracy indices, see Boese 2019; Treir and Jackman 2008; Munck and Verkuilen 2002). In the list of weaknesses, three are critical to limiting the suitability of Polity as a measure of post-conflict progress.

First, Polity has, as a part of its scoring criteria, aspects that capture civil conflict. For instance, in the guidelines for coding the sub-dimensions of a regime's regulation (PARREG) and competitiveness (PARCOMP) of participation, Marshall et al (2017, 75) states that ``[a] polity is coded [as an instance of factional competition] if democratic elections are held in an environment of persistent and widespread civil unrest (rebellion, revolution, and/or ethnic conflict).'' Likewise, when coding whether polities are undergoing ``persistent overt coercion'', Marshall et al (2017, 77) writes that ``[a] polity is coded here if elections are deemed to be “unfair” because ... democratic elections are held in an environment of a persistent, yet largely ineffective or waning, civil violence or ethnic conflict.'' Unlike violence which takes place during an electoral cycle, civil conflict is neither a constitutive dimension of democracy nor autocracy because its occurrence does not reveal the nature of a state's regime by definition. Its inclusion as a coding criterion incorporates irrelevant dimensions into the measurement, and more importantly, hinders an empirical study of the relationship between conflict and regime. Stated in statistical terms, what this coding rule does is introduce the same empirical evidence into the left and the right hand sides of a model. To some extent, since these coding rules confine conflict states to the middle of the Polity scores, the consistent correlation between anocracies and violence, or the so-called murder in the middle, might therefore be an artefact of such coding decisions (Boese 2019). In one study, the inverse U-shaped relationship between conflict and regime in fact disappears once Polity's aggregation method is changed from summation to the weighted average and when measurement error is incorporated into the model (Trier and Jackman 2008).

Second, because of Polity's weighting and aggregation decisions, substantive interpretation can be hampered and a wide variation of countries can be grouped within one regime score (Boese 2019; Trier and Jackman 2008; Munck and Verkuilen 2002). Overall, three dimensions, namely, executive recruitment; executive constraints; and political competition, make up each regime type in the Polity project. Each of these dimensions is, in turn, constituted by a different number of sub-dimensions and their nominal categories. To compute the final democracy and autocracy scores, the Polity project sums up the sub-dimensions along with the weights that are assigned to the majority of these nominal categories. Figure~\ref{fig1} on page 5 summarizes the subdimensions, the categories and the scale weights. These weights are an issue because they create problems for those who seek to glean substantive interpretations from a given Polity score. When both the sub-dimensions and the weights are simply aggregated through summation, every point carried by these weights become valued at the same level as the sub-dimensions. Yet, because the coding manual neither explains the rationale for the weight assignment nor provides a justification for the assigned values, there is no clear way to make sense of the substantive meaning of a final score. Moreover, because these weights increase the number of configurations that can make up the final score, each given level along the Polity index thus comprises a great range of polities. The result is a coarsening of the information which each Polity score conveys. Polity is therefore unsuitable for capturing post-conflict democratic progress since fine differences matter in differentiating among complex political environments and in informing policy makers on important issues of resource allocation.

\begin{figure} [h!]

\caption{Dimensions and Scale Weights in Polity}
\label{fig1}% 
\begin{center} 
\small
 \includegraphics[width=\linewidth]{Boese_Table.jpg}
{\footnotesize Source: Boese (2019, 103)}

\end{center} 
\end{figure}

Third and finally, Polity is an inadequate measure of post-conflict progress because it largely omits an important dimension of democracy, that is, participation (Boese 2019; Munck and Verkuilen 2002). According to the coding manual, participation is operationalized in two ways. The first is ``regulation of participation'', which identifies the extent to which ``there are binding rules on when, whether, and how political preferences are expressed.'' Countries are assigned to different nominal categories based on the extent of factionalism and the degree to which groups are excluded from political involvement (Marshall et al. 2017, 26). The second is ``competitiveness of participation'', which specifies the extent to which diverse opinions are tolerated. Countries are classified under less competitive categories if states and political parties act on repression, exclusion or parochialism (Marshall et al. 2017, 26-7). These coding rules make clear that what is really coded in Polity is exclusion and tolerance, and not participation, which is taken by democracy scholars to mean the extent of inclusion or the degree to which citizens may hold politicians accountable (Held 2006; Dahl 1998). The ability to measure inclusion rather than exclusion is particularly crucial for measuring post-conflict democratic progress. Conflicts oftentimes break out because of exclusive politics; hence, the design of such institutions (e.g. political systems, electoral designs, and constitutions) that influence the extent of inclusivity is a huge matter of contention in post-conflict politics. Without measures capable of picking up on such advancement, Polity leaves out an important assessor of post-conflict democratic peace.
% Other issues with polity: Nominal; Equal weighting (problem of equivalence); Redundancy due to lack of clarity in the levels of abstraction/dimensions; Substantive interpretations of the 0 category

\subsection*{Varieties of Democracy (VDem) and the Problem with the Latent Variable Approach}

The Electoral Democracy Index (EDI) measures the extent to which a country resembles an ideal electoral democracy. By electoral democracy, the Varieties of Democracy (VDem) project is referring to Robert Dahl's (1998) conceptualization of polyarchy, which has, as necessary conditions, six key political institutions: (1) elected officials; (2) free, fair, and frequent elections; (3) freedom of expression; (4) alternative sources of information; (5) associational autonomy; and (6) inclusive citizenship. EDI collapses the third and the fourth dimensions into one, and thus, is constituted by a total of five attributes. Out of the existing democracy indices, the EDI receives a particularly high praise because its measurement is built upon a clear definition; its aggregation rule is clearly outlined and justified; inter-coder reliability is well taken into account; and its use of Bayesian updating allows the index to capture new information that may vary over time (Boese 2019). For the purpose of this project, these improvements are particularly crucial because they resolve the above issues that constrain Polity's suitability as a measure of post-conflict democracy. Unnecessary variables like civil conflict are left outside of EDI's measurement, and both dimensions -- competition and participation -- are clearly represented. Moreover, because VDem's aggregation rules are well-specified, every unit change along the interval is substantively meaningful and interpretation becomes less of an issue.

However, for all its advantages, an issue remains with the use of the latent variable approach in the construction of the EDI. The latent variable approach is oftentimes chosen for its ability to simulate uncertainty and weigh indicators using a data-driven process (Smith and Spaniel 2020; Treir and Jackman 2008). Accounting for uncertainty, and hence, measurement error, is particularly important when the latent variable is to be used as an independent variable because statistical models do not consider errors on the left hand side of the model. A consequence of such omission is the potential biasing of parameter estimates, which in turn, lead scholars to arrive at the wrong conclusions (Trier and Jackman 2008). The simulation of uncertainty, furthermore, allows scholars to quantify the level of confidence which prevents conclusions from being over-stated. Separately, the data-driven approach to estimating indicator weights can also be advantageous because it replaces the need to assign arbitrary weights and instead lets the data reveal relationships that exist within themselves. These benefits notwithstanding, the latent variable approach is inappropriate when the variable at hand is one with non-compensatory dimensions (Wuttke, Schimpf and Schoen 2020).

A reason for the incompatibility between the latent variable approach and necessary conditions is due to the aggregation rule -- averaging -- that is typical of the method. The final outcome is a weighted average of all the indicators, which in the EDI, occurs at the level of aggregation from the indicator level to the conceptual attributes. The sole exception is the dimension of inclusive citizenship, which includes only the extent of suffrage. An implication of applying averages is that higher scores on one indicator can be used to compensate for lower scores on the others because it is the mean value that determines the final score (Goertz 2020; Wuttke, Schimpf and Schoen 2020). This effect of substitutability among indicators leads to concept-measure inconsistency when dimensions are conceptualized as necessary conditions because indicators are not supposed to be substitutable. In other words, a unit should drop out as an instance of the overall concept as soon as it fails to fulfil at least one of the necessary dimensions. When it comes to analyzing post-conflict democratic progress, the key constitutive and non-substitutable dimension is non-violence.

The bases for asserting non-violence as a necessary attribute of democracy rest on both conceptual and theoretical grounds. I elaborate on Dahl's conceptualization of democracy here to stay consistent with VDem's conceptual basis. Dahl (1998, 49) insists that a democracy may only be considered as such if it guarantees rights to its citizens \emph{and} effectively acts to enforce such rights. Rights are central because they grant individuals various forms of freedoms and allow individuals to ``participate fully in determining the conduct of the government'' or defend oneself from a government's abuse (Dahl 1998, 52). Crucially, the democratic tools or institutions that facilitate the exercise of such rights must as such be non-violent; otherwise, the conceptual bases of democracy, such as free and fair elections, cannot be ``free'' (Peh and Goertz 2020; Norris 2014; Schedler 2003; Elklit and Svensson 1997; Schmitter and Karl 1991). Hence, non-violence has to be treated as a necessary dimension for any efforts at conceptualizing or measuring democracy to be complete. The language used within the peacebuilding circle also evidences the expectation of a theoretical relationship between democracy and peace. In the words of Boutros Boutros-Ghali (1996, 8), a former Secretary-General of the UN, ``Democracy within States ... fosters the evolution of the social contract upon which lasting peace can be built.'' Furthermore, some scholars maintain that because democracy and its tools ``provide insitutional mechanisms and legitimacy,'' it is a necessary condition ``without which peace building in a post-war state may not materialize (Joshi 2010, 829).'' Ergo, there exists an assumption that democracy has the potential to cause or promote peace. For democracy to exert such causal powers, conceptualizations of democracy must contain the attributes that are capable of motivating peace (see Goertz 2006); insofar as the non-violence aspect of democracy has the potential to be one such attribute, it must be built in as a constitutive component of democracy (see Goertz 2006).
%Most scholars would agree that democracy is a non-violent means of governance, either by definition or when compared with other forms of governance, at least in intra-state settings (Coppedge et al. 2020; Norris 2014; Davenport 2007; 1997; Diamond 2002; Linz and Stepan 1996; Schmitter and Karl 1991).

What is frequently left unsaid in either the democracy or the conflict literature is the relationship between violence and non-violence (see Campbell, Findley and Kikuta 2017 for an exception). When weighted averages are used, high levels of non-violence are allowed to make up for the presence of varying degrees of violence. Yet, one would hesitate to agree that a country is in a state of non-violence even if the levels of violence is somewhat moderate. Here, I draw on two examples, electoral violence in the Republic of Congo (ROC) and Ivory Coast, to show how EDI levels may show little to no change despite extensive electoral violence. Figure~\ref{fig2} shows the EDI levels of these countries over time. According to Dorina Bekoe (2012, 7),

\begin{quote}
\small
[In ROC,] Even after the election was rerun in November 1993 ... violence ensured. As many as 2,000 people were killed during November 1993 to January 1994 (U.S. State Dept. 1995; United Nations 1997) ... [Four years later, in May 1997, during] the six months of violence until [the leader of an opposition coalition, Denis] Sassou-Nguesso seized the presidential palace, about 15,000 died (Bazenguissa-Gangaa 1999, 39-42; De Beer and Cornwell 1999). Over the next two years, 20,000 more lives were lost (Polity IV 2010). Likewise, the four months of postelection violence in Côte d'Ivoire [Ivory Coast] in 2010 and 2011 broke down along the same ethnic lines and issues that caused the civil war in 2002. Many feared that ... Côte d'Ivoire would return to full-blown war. Even so, wth some three thousand dead and one million displaced ... one may argue that civil war war returned to Côte d'Ivoire in 2011.
\end{quote}

\begin{figure} [h!]

\caption{EDI levels for Ivory Coast and Republic of Congo}
\label{fig2}% 
\begin{center} 
\small
 \includegraphics[width=\linewidth]{EDI_CotedIvoite_Congo.jpg}
 
{\footnotesize Source: V-Dem Institute (2020)}

\end{center} 
\end{figure}


A comparison between Bekoe's (2012) description of the situations in ROC and Ivory Coast and the trends in figure~\ref{fig2} evidences a limitation of using weighted averages. In 1993 (the first dashed line), ROC's EDI fell by an insignificant amount and the country's level of democracy hovered at around the level of 0.43 until the next major episode of violence in 1997 (the second dashed line). Worse still, Ivory Coast's level of democracy sustained an upward trend from 0.48 in 2010 (black line) despite the war-like conditions in the aftermath of the country's election. Thus, by adopting the latent variable approach, EDI falls short when accounting for the balance between violence and non-violence. Conceptually, violence that takes place during electoral cycles should impose especially huge penalities on a country's democracy score because free and fair elections, as mentioned above, are fundamental to the functioning and the conceptualization of democracy. Pragmatically, the difference between de jure and de facto elections also most often form the foundation by which scholars differentiate between democratic and non-democratic regimes (Luhrmann et al 2018; O'Donnell 1996; Schumpeter 1976[2003]). Hence, for non-violence to be taken seriously, the occurrence of violence has to compromise country's democracy levels, especially if it pervades democratic institutions. In this way, only countries which succeed in pursuing meaningful or quality non-violence may be identified as clear members of democratic progress. In the next section, I show how the EDI can be reconstituted to meet the purpose of measuring post-conflict democracy levels and progress.
%fully develop the issue with the latent variable approach - new section?

\section*{Constructing the Post-Conflict Democracy Index (PCDI)}

This section introduces the Post-Conflict Democracy Index (PCDI), a measurement tool that facilitates assessments on a country's democracy levels in post-conflict settings and the degree to which these states have advanced toward the goal of democracy. As the preceding discussion shows, democracy and especially its aspect of non-violence are highly valued by peacebuilding scholars and practitioners. However, existing democracy measures are either inappropriate or fall short of capturing the quality of non-violence. The PCDI addresses the shortfalls by building upon VDem's Electoral Democracy Index and changing, specifically, its aggregation rule.

Figure~\ref{fig3} shows the conceptualization of PCDI. The five dimensions at the second level of the figure represent the five institutions that compose democracy, as defined by the VDem project. Except for suffrage, all dimensions are constituted by a variety of indicators, which in VDem's methodology, are weighted using the Bayesian factor analysis and then averaged to compute a single index score for each country in a given year. Rather than weighted averages, the PCDI first separates the indicators for three of the dimensions, namely, clean elections index, freedom of association, and freedom of expression, into violence and non-violence. Sticking to the conventions of conflict studies, violence refers to direct violence (Galtung 1967), which means, the threat or actual use of force. Figure~\ref{fig3} illustrates this separation at a level below the dimensions level. Although the figure depicts such separation only for the clean elections index, the same procedure applies to all three attributes that have their boxes marked in red. The elected officials index is left out of this step because the indicators constituting this dimension do not embody characteristics of violence or nonviolence. Conversely, the three dimensions that are marked in red make up a common theme, and that is, ``meaningful'' elections (Coppedge et al. 2020, 31; Diamond 2002) or elections that promote political participation and that are free from any form of physical coercion or manipulation (Schedler 2002; Schmitter and Karl 1991).

\begin{figure} [h!]

\caption{PCDI Concept Figure}
\label{fig3}% 
\begin{center} 
\small
 \includegraphics[width=\linewidth]{EDI_Re.jpg}
 
\footnotesize{where \emph{I} refers to indicators}

{\footnotesize Dashed line: Constitutive relationship}

{\footnotesize Arrow: Indicator relationship}

{\footnotesize * : Logical AND}

\end{center} 
\end{figure}

At the lowest level of figure~\ref{fig3} lies the indicator level. Due to the relatively numerous indicators that compose each dimension, all indicators are represented by \emph{I}. In PCDI, as shown in the figure, these indicators are summed because the highest level of violence or non-violence is what reveals the extent of each dimension. To note, PCDI follows VDem's coding on all the indicators. For instance, acts of intimidation by the state during elections are coded between 0 and 4 where 0 indicates high levels of repression and 4 indicates none at all (Coppedge et al. 2019, 57-8). Hence, the so-called highest level of violence has instead the lowest integer. This coding decision makes sense because all indicators would point in a uniform direction: higher values lead to higher levels of democracy. What differs more between PCDI and EDI is the aggregation method at this level. In PCDI, the sums of both violence and non-violence indicators are multiplied. This logic gives both dimensions equal weighting, which follows from the foregoing discussion on the quality of non-violence. In fact, without the intentional elevation of both dimensions, the effect of violence would likely be attenuated since the number of violent indicators is fewer than the non-violent ones across all three relevant conceptual attributes (see the Appendix on the list of indicators).

Finally, to aggregate from the attribute to the conceptual level, some flexibility may be had, and here, I explain how two aggregation methods -- multiplicative or additive -- may be used depending on users' research purpose. The first uses multiplication and measures democracy \emph{levels}. The use of multiplication when measuring democracy levels coheres with Dahl's (1998) argument that each of the institutions is a necessary condition of democracy. The conjoinment of all institutions also ensures that the levels of competition and participation in a country are meaningful, and that elections are occurring in a supportive environment, which makes them more than an institutional facade (Coppedge et al. 2020; Norris 2014; Diamond 2002; Schedler 2002; Linz and Stepan 1996; Karl and Schmitter 1991). The logic of multiplication is suitable for aggregating necessary conditions because they are non-substitutable (Goertz 2020); scoring a 0 on any one criterion effectively makes the unit an irrelevant example for the concept at hand.

Figure~\ref{fig4} illustrates a semantic transformation of the PCDI that is aggregated to show democracy levels in post-conflict settings. The Y-axis represents the degree of membership of a case in relation to the concept of democracy. At the level of 0.0, a case is no where close to resembling a democracy and at the 1.0 level, a case becomes a quintessential example. The X-axis represents a scaled continuum of the levels of democracy following the PCDI from 0 to 1. A notable outcome of elevating the violence and non-violence dimensions in the PCDI is that one can now imagine that countries which act on greater levels of violence would fall closer to 0 while those that act more non-violently will congregate closer to 1 along the continuum. As such, theoretically speaking and taking the arbitrary threshold of a 0.5 score as a point of discussion, we can be sure that cases that line up along the dashed line are more violent than those on the solid black line. Conceptually, we may therefore term this zone that ranges from 0 to 0.5 as ``the twilight zone'' wherein post-conflict countries remain in a zone of non-peace and non-war despite the supposed termination of an armed conflict. Chapter 2 will show how the identification of the twilight zone can be useful as it uses countries that lie within this range to examine how electoral violence in post-conflict states may contribute to conflict escalation in terms of battle-related deaths.

\begin{figure} [h!]

\caption{PCDI (Democracy Levels)}
\label{fig4}% 
\begin{center} 
\small
 \includegraphics[scale=0.6]{PCD_Levels.jpg}
 
{\footnotesize Dashed line: Twilight Zone}

\end{center} 
\end{figure}

To attain the final post-conflict democracy score, PCDI may also be ultimately aggregated through addition. When aggregated this way, the final score produces a reflection of what this chapter has been focusing on so far, and that is, post-conflict democratic \emph{progress}. A higher degree of substitutability is permitted when measuring progress because the intent is to find out how much a country is moving in the direction of democracy. As such, every increment is a point of equal interest regardless of which dimension the improvement is coming from. To highlight, because the dimension of violence and non-violence are aggregated at the level below this final aggregation, the quality of non-violence remains uncompromised. Figure~\ref{fig5} is a semantic transformation of the PCDI when aggregated to depict democratic progress in post-conflict settings. The Y-axis again represents the degree of membership of a case in relation to democratic progress. The closer a case is to 1.0, the more a case is an archetypal case of progress. The X-axis represents a scaled continuum of the degree of democratic progress following the PCDI from 0 to 1. Chapter 3 will elaborate on how this scale can be used as a dependent variable as it seeks to explore how some post-conflict states manage to make significant democratic advancements.

\begin{figure} [h!]

\caption{PCDI (Progress)}
\label{fig5}% 
\begin{center} 
\small
 \includegraphics[scale=0.6]{PCD_Progress.jpg}
\end{center} 
\end{figure}

\section*{Conclusion}

This chapter has surveyed the popular democracy indices that are used in the conflict, and particularly, the peacebuilding literature. More importantly, it builds on the shortfalls of these indices and introduces the Post-Conflict Democracy Index, which seeks to overcome existing limitations and meet the purposes of peacebuilding scholars and practitioners. The biggest contribution of the PCDI is perhaps its flexibility, which allows interested users to assess both the level of democracy and the degree of democratic change. However, more work needs to be done to better specify the index and establish its utility. Therefore, some next steps include the actual construction of the index and comparing it with VDem's Electoral Democracy Index to evaluate whether the proposed index indeed enhances our understanding of democracy in post-conflict contexts.

\section*{References} 

\begin{hangparas}{1cm}{1}

\end{hangparas}

\section*{Appendix}

\begin{figure} [h!]

\caption{PCDI Violent and Non-Violent Indicators}
\begin{center} 
\small
 \includegraphics[width=\linewidth]{PCDI_Appendix.jpg}
 
{\footnotesize Elected Officials Index and Suffrage are not included in the table because all indicators are non-violent}

{\footnotesize EMB: Electoral Management Body}
 
{\footnotesize CSO: Civil Society Organizations}
 
\end{center}
\end{figure}







\end{document}


















