\documentclass [11pt]{article}

\title{Bringing Legitimation Back In: Bridging State Formation and Post-Conflict Democratic Progress}
\author{Kimberly Peh}
\date{\today}

\usepackage{hanging,verbatim,geometry,rotating,graphics,epigraph,afterpage,url,pdfpages,pifont}
\usepackage{etex,tikz,xcolor,verbatim,geometry,afterpage,float}
\usepackage{rotating}
\usepackage{pgfplots}
\usepgflibrary{arrows.meta} % need this for arrow tips
\pgfplotsset{width=10cm,compat=1.14}
\usetikzlibrary{datavisualization}
\usepgfplotslibrary{statistics} 
\usepackage{Times}

\geometry{height=8in, width=5in}

\usepackage[utf8]{inputenc}

\setlength{\epigraphwidth}{.8\textwidth} \setlength{\epigraphrule}{0pt}

\urlstyle{same}

\lefthyphenmin=2
\righthyphenmin=3

\brokenpenalty=10000 % No broken words across columns/pages
\widowpenalty=10000 % No widows at bottom of page
\clubpenalty=10000 % No orphans at top of page

\begin{document}
\maketitle

\section*{Introduction} 

Under what conditions do post-conflict states achieve democratic progress? Since the end of the Cold War, practitioners and scholars of peacebuilding have touted both peace and democracy as the twin goals of post-conflict peacebuilding (Manning and Smith 2016; Fortna and Howard 2008; Hippler 2008; Reilly 2008; Carothers 2007; Boutros-Ghali 1996). However, while a consensus may be reached on the goals, a debate persists on the `best' way by which these goals may be achieved (Walters 2015; Jarstad and Sisk 2008; Carothers 2007; Paris 2004).

Two main approaches characterize the core of the debate. The first may be called the sequentialist paradigm. Leading this school of thought are scholars who, after reflecting on the effects of external interventions in inter- and intra-state wars, argue that early democratization efforts are counterproductive to the pursuit of either peace or democracy (e.g. Reilly 2017; Brancati and Snyder 2012; Grimm 2008; Mansfield and Snyder 2005; Paris 2004; Zakaria 1997). These ``early democratization efforts'' refer most often to the conduct of elections, which these scholars emphasize is a fundamentally divisive form of politics that may not only entrench social and political cleavages but also ease the mobilization of resources for the renewal of war (Letsa 2017; Paris 2004). As such, scholars advocate for the state-building approach, which prioritzes the building of political, economic and secuirty institutions before the implementation of democratic tools. %Here, the assumption is that a strong state is necessary to mitigate the conflict-inducing effects of democratization (Grimm and Merkel 2008).

The second paradigm may be termed the gradualist approach (Carothers 2007). This framework espouses the concurrent implementation of both state- and democracy-building tools. According to Thomas Carothers (2007), the sequentialist framework rests on shaky theoretical foundations and assumptions which do not justify the sequentialists' skepticism over the value of promoting democracy early on in post-conflict states. Others show that democratic behaviors can be habituated (e.g. Lindberg 2003), and hence, the early conduct of such democratic institutions as elections can lay the ground for democratization in the future.

The gradualist approach is also that which is most commonly promoted and practiced by the United Nations (UN) and the United States. Across the range from conservatives to liberals and from doves to hawks, US presidents have used or endorsed the merits of democracy as a reason for intervention or a means of conflict resolution (Hippler 2008). Asserting its neutral stand, the UN cites the seemingly peaceful nature of democracies -- most often supported by the Democratic Peace theorists (e.g. Doyle 2005; Russett et al. 1995; Maoz and Russett 1993) -- as a justification for promoting democracy in post-conflict settings (Boutros-Ghali 1992). Due to the enthusiastic promotion of democracy, virtually all post-conflict countries today hold elections as a way to signify the end of a conflict and the time between the signing of an agreement and the conduct of an election reduced by approximately half (from 5.6 years to 2.7 years) after the end of the Cold War (von Borzyskowski 2019; Brancati and Snyder 2012).

%The prevailing debate casts differences between the approaches as irreconcilable. I argue that the seemingly antithetical nature of these approaches stems from the gradualists' narrow view of what constitutes democratic tools and the sequentialists' shortfall in regard to specifying the quality of institutions and the mechanisms they expect would link institutions to conflict prevention. Putting together the ideas from both frameworks, I advance the logic of democratic state formation, which posits that post-conflict states can attain democratic progress -- shifts towards meaningful elections and citizen participation -- when institutions that constitute a state take on a democratic character. By ``democratic character'', I mean the broad-based nature of politics and the constraints imposed on governments. Specifically, it refers to the availability of platforms (e.g. protests and constitutional courts) through which citizens may legally use to contest a government and evidence of such effort by a government to justify its right to rule in exchange for citizens' support. Notably, the relationship between having a democratic character and democratic progress is not tautological. Many countries in North America and Europe today provide citizens with forums to contest governments and require that parties renew their mandate at least during each electoral cycle; yet, even these countries are not immune to the problem of autocratization or backsliding.
% Theoretical foundation -- Weber, Tilly, Hui, Wood, Huang, Wantchekon, O'Donnell (brown, blue and green zones -- only look at the legal dimension)
% re-work some of these two paragraphs following the change in argument

This chapter largely advances the sequentialist approach by proposing a different way to conceptualizing state formation. Specifically, it brings in the dimension of legitimation to shift the focus away from a capacity-building framework, which places a greater emphasis on monopolizing security forces and that tests political, legal and security institutions independently. Many of the institutions that willl constitute the state formation concept are admittedly familiar, but with this new conceptualization and the use of a different method, Qualitative Comparative Analysis (QCA), this chapter makes two key contributions.

First, it overcomes a critique against the sequentialist approach that highlights its lack of a clear causal relationship with democratic progress, that is, advancement towards greater levels of democracy. This contribution is made possible with the introduction of legitimation because it privileges the legal and international institutions that serve, on the one hand, as sources of justification for a post-conflict state to legitimize its rule. On the other hand, they act to some degrees as constraints upon a state, for instance, by sanctioning the state or subjecting it to legal contestations. To the extent that these institutions influence how a state uses its forces and move the site of contention toward the legal fora, reductions in the occurrences of large-scale violence can move states, however slightly, towards greater levels of democratic progress.

Second, by building one of the democratic tools -- constitution-writing -- into the legitimation dimension and by using QCA, a method which looks for the effect of a conjunction of, rather than independent, causes, this chapter can potentially reduce the size of the rift that presently characterizes the two camps. Furthermore, QCA generates different combinations of conditions that may lead to an outcome; hence, further case analyses of each of the pathways can shed light on whether the issue of sequencing is universally applicable, and the conditions under which each pathway may take place.
%this paper also corrects for some methodological issues in the literature by combining Qualitative Comparative Analysis (QCA) with case study analysis. This chapter's focus on portable mechanisms and theoretical logics can also shed light on policies. If the argument holds true, it lends support to those who call for external interveners to have a more comprehensive plan; provides interveners with a framework to evaluate such a plan; and joins some scholars in urging interveners to focus also on bottom-up paths to democratization (Schmidt 2020; Hayman 2014; Malik 2014; Rao 2013; Wood 2001; Lemay-Hébert 2009; Hippler 2008).
% see the NUS file on post-war peace-building and civil wars

%argument is not entirely new. Scholars have used the state formation lens to study post-conflict peacebuilding (Karim 2020; Sosnowski 2019; Shinoda 2018) and the institutions that I will use to define a state are common explanatory variables (e.g. demobilization and legal fora) in the post-conflict democratization literature. Still, the democratic state formation thesis adds value to the ongoing debate. It synergizes a debate that highlights differences over similarities; logically brings together the presently disparate findings; and provides a clear causal link between institution-building and democratization in post-conflict settings. These contributions are crucial because the state-building paradigm largely tests these variables independently, asserts the necessity of different institutions, and falls short of providing an empirically tested causal chain.

The following section continues the discussion by outlining the problems in the existing literature on post-conflict democratization. The subsequent section then elaborates on the conceptualization of state formation. The fourth section describes the research method -- QCA. The fifth section concludes with some next steps in this research effort.

% a section on why study post-conflict democratic progress?

\section*{Key Issues in Studying Post-Conflict Democratization}

%This section focuses on the shortfalls of the sequentialist approach because the weaknesses of the gradualist approach are well-documented in the literature (see for example, Reilly 2017; Flores and Nooruddin 2012; Jarstad and Sisk 2008; Paris 2004; Kumar 1998; Zakaria 1997). Moreover, the sequentialist approach was introduced precisely in reaction to the democratization programs, such as post-conflict elections and rebel-to-party transformations, which have undermined the viability of peace and democracy in some post-conflict countries (Reilly 2017; Lyons 2016; Ishiyama 2016; Brancati and Snyder 2012; Flores and Noorrudin 2012). Hence, this chapter departs from the foundations of the state-building literature and seeks, in particular, to fill two gaps that remain to be addressed.

% An interesting characteristic of the literature on post-conflict peacebuilding is that the goals of democracy and peace are often studied separately. Scholars who are most concerned with the duration of post-conflict peace usually focus on the risk or likelihood of conflict recurrence (e.g. Letsa 2017; Walter 2015; 2004; Brancati and Snyder 2012), while others narrow in on the prospect of democratic transitions in post-conflict states (e.g. Lyons 2016; Joshi 2010; Gurses and Mason 2008; Wantchekon 2004; Wood 2001).

%When both are assessed in conjunction, the questions asked regard the extent to which democratization efforts hinder peace, and whether the means of conflict management hamper the possibility of future democratization. The answer, frequently and pessimistically, is that the pursuit of one tends to generate unintended repercussions on the other (see Jarstad and Sisk 2008, and specifically Höglund 2008). While this research angle, which emphasizes trade-offs, has provided valuable insights to policymakers and has advanced post-conflict research, it has also as a consequence, a blind spot where potential synergies are concerned.

%Notwithstanding its strengths, this literature also suffers from several theoretical and methological limitations, of which, three will be discussed in this section. The first and second relate to the arguments put forth by proponents of the state- or the institution-building paradigm. A key assumption underlying this argument is that a strong state, composed of strong institutions, is a necessary condition for democracy (Brancati and Snyder 2012; Grimm and Merkel 2008; Hippler 2008; Paris 2004; Kumar 1998; Boutros-Ghali 1996, 8). Some scholars cite, for example, Larry Diamond (2006: 94), who writes that `Before a country can have a democratic state, it must first have a state.'' Others quote Juan Linz and Alfred Stepan (1996, 17), who famously state that `without a state, no modern democracy is possible.'' Still others refer to Samuel Huntington (1968: 4), who highlights the relationship between violence and the ``slow development of political institutions.''

%In this research agenda, state-building is therefore the priority that is to precede democratization (hence the state-building approach is also known as the sequentialist approach), and it is, importantly, a necessary condition for preventing democratic efforts (specifically, elections) from undermining peace and democracy (Reilly 2017; de Zeeuw 2008). Yet, research thus far has not truly tested the necessity of the state as a precondition. Statistical analyses using hazard models and logit regression are the norm, and they test rather the extent to which institutions such as demobilization programs, power-sharing agreements, and rule of law, influence recurrence or democratic outcomes than the degree of constraint these institutions impose on the possibility of attaining these goals (see Goertz and Mahoney 2012).

%Dawn Brancati and Jack Snyder (2012, 828-9) suggest an alternative to the necessary condition perspective. According to them, several individually necessary conditions may be jointly sufficient in facilitating the success of post-conflict elections. They write,

%\begin{quote}
%\small
%Early elections are less risky when one side has won a decisive military victory since the losing side lacks the ability to return to fighting if it fares poorly in the election. In the absence of a decisive victory, demobilization of one or both sides, or their integration into a new army, can mitigate the risk of early elections. Successful demobilization is a complicated and lengthy process \dots Strong bureaucratic institutions and generous financing are needed to facilitate demobilization. The development of robust administrative institutions, international peacekeeping, and economic development can also facilitate this process (Doyle and Sambanis 2006; Fortna 2008b). Of these factors, peacekeepers are more likely to be in place if elections are held soon after war ends, whereas the other factors are more likely to be favorable if elections are held later.
%\end{quote}

%Carrie Manning and Ian Smith (2016, 973) make clear the gap between such a theoretical view and the methodological practice when they state that ``[w]hile most statistical treatments assume that independent variables are largely independent of one another, we know that empirically the variables we identify are linked to one another in complex ways, and the impact of one factor is likely to be contingent on the presence of others.'' Indeed, despite recognizing the possibility of causal complexity (see Goertz 2017, Chapter 3), Dawn and Snyder (2012) proceeded with a logit analysis, which analyzes the net effect of each variable, \emph{ceteris paribus}. Overall, therefore, there is a theoretical and methodological gap in the literature wherein neither the necessity nor the jointly sufficient expectation of the explanatory conditions is tested.

The first weakness in the broad literature on post-conflict democratization inheres in scholars' operationalization of ``post-conflict democratization''. The majority of existing scholarship on the topic uses data from the Polity IV project which Chapter 1 has shown to be an inappropriate measure for conflict research (an exception is Fortna and Huang 2012, who used Polity-X in their robustness check. On Polity-X, see Vreeland 2008). Furthermore, to calculate the degree of democratic change in post-conflict countries, scholars primarily take the difference between a country's pre- and post-conflict Polity scores. As Virginia Page Fortna and Reyko Huang (2012, 802) point out, a key problem with this operationalization strategy is that changes to democracy levels might have occurred prior to the onset of some explanatory variables, such as war outcomes, which violate a fundamental guideline of causal inference: temporal priority (Coppedge 2012).

The second issue is the sequentialist approach's lack of attention to the causal chain, specifically, the mechanisms that link the explanatory conditions to the outcome of democratization. While the sequentialist paradigm is born largely out of case studies, these cases were analyzed mostly to evaluate and criticize the democratization approach (e.g. Mansfield and Snyder 2005; Paris 2004). More recent studies took on a statistical turn and mechanisms that are proposed remain as such theoretical propositions. The vagueness regarding the causal chain led Thomas Carothers (2007), a champion of the gradualist approach, to publish a scathing response to the sequentialists that point to the potential for strong states to take on an autocratic character. Carothers (2007) reasons that absent constraints on potential autocrats, strong state structures may be usurped by leaders to silence dissent and competition. This argument finds support in the state strength literature which implicitly perceives states with democratic regimes to be weak (Vu 2010; Skocpol 1985).

% Stops at FPE, ~ progress <--> ST/LT effects (e.g. power sharing) - leaving it out because the problem is more pertinent to the second chapter

\section*{Bringing Legitimation Back In}

Currently, proponents of state-building mostly adopt an institutionalist approach, which privileges a ``state's ability to provide security'', its administrative capability and the ``ability of the state apparatus to affirm its authority'' (Lemay-Hébert 2009, 23). Unsurprisingly, the institutions that are tabled for consideration include demobilization programs, the strengthening of a police force, legal frameworks, a functioning bureaucracy, economic reforms and the like (Brancati and Snyder 2012; Paris 2004; Wantchekon 2004) -- institutions that concern ``the capabilities of the state to secure its grip on society (Lemay-Hébert 2009, 23).'' The problem is that such a `strong' state is as, if not more, likely to exhibit autocratic tendencies. 
%This focus on a state's authority and capacity finds its relevance in much of the state formation literature, which studies the two dimensions jointly under the notion of state strength (Kurtz 2013; Vu 2010; Mann 2008; Centeno 2002; Skocpol 1985; Tilly 1985).

To correct for the problem, a theory of democratic progress from the state-building perspective requires a deeper discussion of how the development of a strong state can motivate a causal chain that leads to a democratic outcome. To do so, this section begins with assessing the conceptual basis that informs the sequentialist view of state-building. It contends that the conceptualization of the `state' is missing a key dimension, legitimacy. By bringing legitimacy back into the definition of the state, and hence bringing the process of its attainment back into the conceptualization of state formation, this section shows that there are viable causal links between state-building and democratic progress.

\subsection*{Conceptualizing the State}

A theoretical basis that aligns with the sequentialist view of state-building is the understanding of a state as an entity that monopolizes force (Tilly 1992; 1985, 171; Skocpol 1979; Hobbes 1651[1996]). This view essentially truncates Max Weber's (1965, 1) definition of a state as “a human community that (successfully) claims the monopoly of the legitimate use of physical force within a given territory” into that which exercises the monopoly of force (Vu 2010, 150). The concern as such lies with the coercive capacities of a state, which scholars argue makes possible societal acceptance or law and order -- features of the ``legitimate'' dimension in Weber's definition -- due to the consequent ability to provide security or public goods (Kurtz 2013; Tilly 1985, 172; Elias 1982; Hobbes 1651[1996]). Charles Tilly (1985, 171-2) states,

\begin{quote}
\small
The distinction between ``legitimate'' and ``illegitimate'' force, furthermore, makes no difference \dots

Legitimacy is the probability that other authorities will act to confirm the decisions of a given authority. Other authorities, I would add, are much more likely to confirm the decisions of a challenged authority that controls substantial force; not only fear of retaliation, but also desire to maintain a stable environment recommend that general rule. \dots A tendency to monopolies (\emph{sic}) the means of violence makes a government's claim to provide protection, in either the comforting or the ominous sense of the word, more credible and more difficult to resist.
\end{quote}

A proper read of Weber's definition evidences that the coercion-centric interpretation of a state is incomplete, and it misconstrues the relationship between the processes that underlie state formation. Weber (1965, 1) writes,

\begin{quote}
\small
In the past, the most varied institutions \dots have known the use of physical force as quite normal. \dots [A]t the present time, the right to use physical force is ascribed to other institutions or to individuals only to the extent to which the state permits it. The state is considered the sole source of the `right' to use violence. Hence, `politics' for us means striving to share power or striving to influence the distribution of power, either among states or among groups within a state.

Like the political institutions historically preceding it, the state is a relation of men dominating men, a relation supported by means of legitimate (i.e. considered to be legitimate) violence. If the state is to exist, the dominated must obey the authority claimed by the powers that be. When and why do men obey? Upon what inner justifications and upon what external means does this domination rest?
\end{quote}

From the excerpt, we see that Weber does not define a state by its ability to monopolize force; rather, what a state monopolizes is the `right' to use force, or what he succinctly calls `legitimate violence'. Accordingly, political contention with and over the identity of a state is about contesting who gets to exercise and share this right, and not about who possesses force per se. Without this dimension of rights, one's conceptualization is, thus, incomplete.

Relatedly, because a state is about having and monopolizing legitimate force, legitimation is an active part of a state's (re)constitution. It is worth highlighting that the paper entitled ``Politics as a Vocation'', in which Weber defines and discusses the state, is precisely about legitimation strategies for justifying a state's right to command force and hence, obedience. Weber's emphasis on legitimation comes as no surprise since legitimacy is key in his conceptualization of a state. The attention toward legitimacy, and hence legitimation, thus differentiates Weber's proposition from Tilly's (1985). In Tilly's, legitimation is inconsequential because the monopoly of force brings about legitimacy; in Weber's, legitimation runs alongside strategies to acquire and monopolize force because legitimacy is a defining feature of being a state.

The histories of Asia, Europe, and Latin America attest to the centrality of legitimacy or legitimation in the process of state formation. Victoria Hui (2005, 48) finds that even the most aggressive Chinese rulers acceded to popular bargains to ensure that support for wars continued through an ongoing supply of food, taxes and recruits. In Europe, Charles Tilly (1992, 26) observes that it is ``inevitable'' for elites to extend rights to various groups to create buy-in and ``extract still more resources'' during periods of war preparation. In post-communist Europe, Latin America and East Asia, weak legitimacy is sometimes cited as a reason for the weakness of central states (North, Wallis, Webb and Weingast 2012; Ganev 2005; Centeno 2002).

Engagement of legitimation activities (e.g. rebel diplomacy and public goods provision) by clans, warlords, and rebels has also been described as attempts by non-state armed groups (NSAGs) to exhibit state-like qualities (Callimachi 2018; Arjona 2017; 2016; Huang 2016; Lara 2014; Kilcullen 2013). Notably, the failure of some of these NSAGs to attain legitimacy shows that legitimation does not guarantee legitimacy (see Gerschewski 2018 for a discussion on their differences). But when legitimation efforts work, they can influence civil war dynamics and outcomes by changing citizens' or international actors' willingness to recognize these groups' ``viability and authority'' (Jo 2015, 14). Recognition, in turn, can affect materiel support and hence, the balance of power or the extent to which defection or whistle-blowing occurs (Arjona 2016; Wood 2003; Kalyvas 2000). Legitimation, as such, is a key part of state formation, and legitimacy is an important quality for actors that seek to be a state.

Based on this discussion, a complete conceptualization of a state (see figure ~\ref{fig1}) thus comprises three dimensions: the fundamental existence of a physical force (this is a non-trivial dimension especially in a post-conflict setting that this section will explain later), its monopoly, and its legitimate use. It is only when an entity fulfills all three conditions (i.e. exists within the gray zone) that it may be called a state. The next subsection uses this conceptualization as a point of departure to include legitimation in the description of state formation.

\begin{figure} [h!]

\caption{Conceptualization of the ``State''}
\label{fig1}% 
\begin{center} 
\small
\includegraphics[width=\linewidth]{Venn.jpg}
 
\end{center}
\end{figure}

\subsection*{Conceptualizing State Formation} % non-violence

The concept of state formation, defined as the process of becoming or sustaining an entity as a state, builds on the three dimensions in  figure ~\ref{fig1}. Hence, it includes the building of a force, its monopoly and its legitimation. To highlight, because the context that concerns this study is the post-conflict setting, the focus is not on wars or war preparation efforts but their aftermath. In some cases, a state may remain intact but in some others, such as newly independent countries, a state may need to be built somewhat from scratch.

In post-conflict settings, the need to build a force is therefore non-trivial. Especially in conflicts that are protracted or that end up with neither side having the means to eliminate the other, one or both sides that go on to compose a state may lack the resources or manpower to provide public goods, such as security. Hence, inasmuch as the possession of force seems fundamental to the universe of states, the process of (re)building a state requires the effort to put together a force.

The monopolization of a force pertains, then, to outright military victories or the disarming of NSAGs. Short of military victories, demobilization in post-conflict settings may involve, for instance, the co-optation of rebels into the national military or the police forces, or the reintegration of rebels into the society. The goal of this feature is to make sure that the state is the only authority that wields force.

However, sticking close to Weber's definition, the monopolization of force does not mean that the national military and police forces must be the only organizations that retain force. Because a state decides who gets to share the use of force, NSAGs like clans, criminal groups, and paramilitaries can continue to exist, especially if they are sanctioned or tolerated by the state -- formally or informally. Evidence of such state behavior is rampant in different parts of Latin America and Southeast Asia, and had led scholars to theorize the overlaps between a state and its obverse, perhaps crudely known as the non-state (Trejo and Ley 2020; Migdal 1994).

Finally, legitimation refers to the process of justifying an actor's right to use force. Particularly in the post-Cold War post-conflict settings, this `right' comes from the action of elites (as Weber's attention toward legitimation strategies suggest), which encompass generally the state, rebel leaders, and international interveners like the United Nations (UN), regional organizations and major powers. Empirically, we most often see this right being enshrined in constitutions that address to varying degrees the underlying sources of a conflict, and the UN's recognition of a country as a member of the community of States.

While UN recognition clearly indicates a state's status or relationship vis-à-vis other international actors, constitutions signify to different degrees a state's relationship with its society (the so-called non-state actors in a country). In some cases, rather than justifying to societies their right to rule within a given territory, domestic elites may use constitutions primarily to gain international recognition. This observation coheres with the critiques of the liberal peacebuilding framework, which posit that negotiations over constitutions are hardly democratic and that constitutions are more often tools to get the international community to conclude that a conflict has terminated. That said, constitutions legalize a state's identity and powers in relation to a society, and as the definition of legitimation goes, citizen recognition is not required.

Figure ~\ref{fig2} (in page 13) illustrates the basic framework of `state formation'.

\begin{sidewaysfigure}[ht]
\caption{Basic Framework of State Formation}
\label{fig2}
\begin{center} 
\includegraphics[width=\textwidth]{Concept.jpg}

{\footnotesize Double dashed line: Ontological relationship}

{\footnotesize Arrow: Indicator of}

{\footnotesize * : Multiplication (indicates the relationship of necessity among dimensions)}

{\footnotesize + : Addition (indicates the relationship of substitutability among indicators)}
  
\end{center}
\end{sidewaysfigure}

The dimension level in figure ~\ref{fig2}, that is, the second level of the basic framework, is aggregated using multiplication because all dimensions, as discussed above, are necessary components of a state. This choice of aggregation implies that while states may perform differently on each of the dimensions, the absence of one (i.e. scoring zero on one dimension) removes a case from the population of states. \emph{De facto} states, for example, are thus excluded because they fall short, by definition, on at least one of the attributes.

%Empirically, a concern may arise with the use of this strict criterion because the weakness of states in post-conflict settings may lead them to devote minimal effort or resources to building states. By implication, many post-conflict states may be excluded. This section addresses this issues when it elaborates on the indicator level (the lowest level of the basic framework).

The indicator level (the lowest level) of the basic framework shows how each dimension is operationalized at the empirical level. The indicators chosen to measure the monopoly and the legitimation dimensions are somewhat self-explanatory. Since monopoly of force pertains to a state's disarmament of actors whom it chooses to not authorize the use of force, this feature can be observed when such armed actors are demobilized or when states attain miltary victories. These indicators can be gathered, more specifically, from observing the disbanding of a rebel group, and conflict termination outcomes.

Legitimation, at least in post-conflict contexts, concerns a state's (re)formalization of its position in the international or the domestic settings. Hence, international recognition and constitutions are arguably appropriate indicators. That said, the substitutability of one measure for the other can be unsatisfactory. However, given that a constitution may be (re)drafted only sometime after the internationally recognized date of conflict termination, if at all, international recognition can be helpful in indicating at least some progress in (re)forming the state. To operationalize these indicators, measures include changes to constitutions that indicate the binding effect of provisions on both state and non-state actors (Elkins et al. 2014), and a UN pronouncement that officially recognizes the post-conflict state.

The dimension regarding the building of a state's force is measured, finally, by changes in the size of the military or the police forces, or changes in the level of energy consumption, accounting for population size. Including consumption patterns measure to some extent a state's ability to provide for the needs of a society and the changes in a society's ability to afford consumption. Notably, a more popular indicator is the gross domestic product (GDP). However, GDP is too broad a measure and it may disproportionately capture a state's spending power over the society's.

Changes to the strengths of the security forces relate to changes in a state's ability to provide security. To be sure, there is no guarantee that states will refrain from deploying the forces to conduct more repressive, and potentially more invisible, forms of violence (Earl 2003). Hence, the constraining effect via multiplication at the dimension level is key; an entity that sustains its position through force cannot be one that is advancing toward becoming or sustaining itself as a state, as defined above. 

This notion of constraint among the dimensions is also the factor that makes imaginable mechanisms that may link state formation with democratic progress. For instance, states may gradually engage in less large-scale violence to avoid international sanctions or censure. Likewise, with a domestic legal framework to contest states, more institutionalized and non-violent means may be adopted by citizens to challenge a state's authority and scope of action. Quoting Joel Migdal (1994), `` it is the legal framework of the state that establishes the limits of autonomy for the associations and activities that make up civil society. If that framework is widely accepted, then the activities of the state and other social groups may be mutually empowering.'' Theoretically, this suggests that iterative experiences with leveraging and contesting over legal frameworks can potentially strengthen non-state actors and importantly, legal institutions, which may, in turn, reinforce their use and effectiveness.

There may be several pathways by which advancements in the process of state formation can lead to progress along a continuum of democracy levels. The next section describes the research method -- QCA -- that facilitates the study of the different pathways.

\section*{Research Design}

Crisp-Set Qualitative Comparative Analysis (QCA) is a method founded upon set theory, which is designed to uncover configurations of conditions that explain an outcome. This method is especially suitable because it seeks to understand how different causal conditions, in this case, from both the sequentialist and the gradualist paradigms, interact with one another to explain an outcome. In contrast, the analysis of multiple interactive terms is oftentimes difficult in and less appropriate through large-N quantitative methods, which are more suited for studying the net effect of each independent variable.

Substantively, the adoption of QCA makes sense as well because post-conflict settings are complex environments. As Carrie Manning and Ian Smith (2016, 973) state, ``[w]hile most statistical treatments assume that independent variables are largely independent of one another, we know that empirically the variables we identify are linked to one another in complex ways, and the impact of one factor is likely to be contingent on the presence of others.'' Yet, the majority of studies in the literature uses hazard and logit models, which tests exactly for the net effect of each explanatory condition.

In this analysis, the other explanatory conditions that will be included are provisions to transform rebel groups into parties; the length of time between conflict termination and the first post-conflict election (FPE); the presence of a peacekeeping operation; conflict duration; and the number of battle-related deaths. No consensus is reached on the effect of any of the variables on democratic progress. However, these international, conflict, or democratization conditions, in addition to the state formation ones, are the most widely-tested in the literature. Data on these conditions can be gathered from the Comparative Constitutions Project (CCP), the National Elections Across Democracy and Autocracy (NELDA) dataset, the Uppsala Conflict Data Program (UCDP), and the UN.

On the dependent variable side, the outcome of interest is democratic progress. The term democratic progress is preferred to democratization to differentiate movements along the democracy continuum from the more complex concept of democratic transition. The study of democratic progress in this chapter may not therefore shed light on transitions toward democracy, taken to be the crossing of a particular level or the achievement of some minimal conditions of democracy. This choice of definition stems from a sense of realism (with a small r). Few post-conflict states may be said to have crossed any threshold of democratic transition, and even those praised as ``beacons of democracy'' (Wantchekon 2004) continue to be shrouded in violence. In El Salvador (Montoya 2018, 4), 

\begin{quote}
\small
``ordinary Salvadorans often described the postwar moment as one of ‘neither war nor peace’ or one even ‘worse than the war’ (Montoya 2007; Moodie 2010: 84). In invoking the war as a constant point of reference against which the present was assessed, such statements called into question the country’s transition to democracy.''
\end{quote}

Similarly, in Guatemala (Knowlton 2017, 140),

\begin{quote}
``the state’s counterinsurgency violence [transformed] into a kind of state-supported violence in which government institutions act at the behest of national and transnational agribusinesses and mining companies to evict Q’eqchi’ from their traditional territories. In many cases the Guatemalan military and the National Civilian Police act as accomplices, accompanying and legitimizing evictions and acts of violence against Q’eqchi’ [a minority group in Guatemala] individuals and communities.''
\end{quote}

Hence, this chapter measures movements towards the direction of democracy, broadly speaking. Data for this measurement will be gathered from the Post-Conflict Democratization Index (PCDI), which is proposed in Chapter 1. The use of PCDI as the source of measurement is an improvement to the problem in the literature, which tends toward using the Polity Index as a measure of democracy. To recapitulate, because Polity includes the occurrence and features of a conflict in its construction of the index, it introduces variables into both the left and right hand sides of a model. More importantly, it does not value adequately the dimension of non-violence, which is key to the definition of democracy.

However, this chapter will not be using the raw values of the PCDI. Due to the elevation of the non-violence feature of democracy and the aggregation of the concept using the logic of weakest link (i.e. multiplication), many post-conflict states score close to 0 on the PCDI scale. Imaginably, differences across states and changes over time will be mostly minimal if even the democracy hopeful, El Salvador, continues to remain in a twilight situation. Thus, all explanatory and outcome variables, including all indicators in the newly conceptualized state formation condition, will be transformed semantically along a scale from 0 to 1 before the QCA analysis. There are two advantages to this transformation beyond the mechanical fact that crisp-set QCA evaluates zeros and ones.

First, it allows the context, post-conflict settings, to be taken into account. When cases are scored closer to 0 or 1 on each condition depending on how scholars understand the concept and the relevant set of real-world cases, scores take on substantive meaning that are grounded within the given context. Second, it prevents the potential negative changes, which can happen in reductions to the size of the military or the police force, to result in negative scores. Having negative scores can make the overall concept of state formation unwieldy and they introduce too much room for different substantive interpretation, especially if the range of scores becomes huge. All conditions will then be dichotomized for the QCA analysis, with cases crossing the threshold of 0.5 being coded as 1, indicating a presence of the dimension, and the rest being coded as 0.

\subsection*{Case Selection}

Because QCA generates conjunctions of conditions (i.e. pathways) that produce the outcome of democratic progress, it is essentially an analysis of `successful' cases. In a literature that hardly studies what works, how conditions interact, and tests the causal mechanisms, this feature of QCA can be a huge contribution because it draws attention to these undertheorized parts of the literature. More importantly, because pathways may be considered `successful', by convention, when they produce the same outcome around 80 percent of the time, QCA allows an analysis of both the mechanisms through analyzing the successful cases, and the scope condition that limits the replicability of success in the rest of the cases.

Following this logic, cases that populate a pathway and that have attained progress will thus be chosen to investigate the mechanisms, and those that failed to attain progress despite the same condition will be used for the study of the scope condition. The parsing out of mechanisms is especially crucial to the ongoing debate between the sequentialists and the gradualists because there may exist sequencing issues that are hidden by the inability of QCA to incorporate temporality. At the same time, knowledge of how different pathways work and the conditions within which they work can be particularly crucial to policymakers and peacebuilding practitioners who carry the burden of allocating scarce resources and building not just states, but lives within these war-torn zones.

\section*{Conclusion}

This chapter engages the debate on post-conflict democratic progress to find out the conditions under which advancements toward democracy may occur. It does so by reconceptualizing the concept of state formation to better specify the arguments that relate state-building with democracy. It also uses QCA to reconcile the divide between the two camps to see how the problem of sequencing may vary across different contexts.

Some next steps would include better specifying the notion of legitimation and its indicators, and the other variables that may conjointly influence the outcome. More thought also needs to be given to theorizing the relationship between the proposed concept of state formation and democratic progress, considering that many states have continued to use non-state armed groups as a means to repress societies -- a means that is often hidden from the sight of the international community or that is less high up on its agenda. Similarly, just because constitutions exist does not mean that societies will recognize or engage with the document. Legal institutions such as courts may also not be independent enough to serve as a proper forum for non-state actors to challenge the state legally. Even civil society organizations themselves may not be `civil' in their means of contestation; hence, even with legitimation strategies, more attention needs to be given to understanding the state-society interaction -- an analysis which can hopefully be done through in the case study analysis.

%Lemay-Hebert 2009 - implications of absence of local legitimacy on the success of external interventions (p. 37 onward)

%Not tautological - legitimation strategy and also, dependent on the balance among the three conditions, and whether the pursuit of legitimacy is temporary, as in China
%And mass mobilization can lead to war/prevent wars from conclusion -- e.g. territorial wars where the vast majority does not see a stake/Letsa
%social acceptance (can go both ways -- evident in the war preparation process, but it is part of the process that ultimately made the state), which is why fora for contestation are key
%causal mechanism standpoint: Non-violence - to balance the coercive dimension; 

%caveat against the romanticization of local peace (Richmond 2009); fallacy of a civil civil society















% leave out in this version the logic and citizen action; focus on the conditions and redefining state formation

\end{document}
