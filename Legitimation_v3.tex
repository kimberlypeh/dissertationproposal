\documentclass [11pt]{article}

\title{Spoiling by Traditional Authorities: The Influence of Democratic Constitution-Making on Electoral Quality}
\author{Kimberly Peh}
\date{\today}

\usepackage{hanging,verbatim,geometry,rotating,graphics,epigraph,afterpage,url,pdfpages,pifont}
\usepackage{etex,tikz,xcolor,verbatim,geometry,afterpage,float}
\usepackage{rotating}
\usepackage{pgfplots}
\usepgflibrary{arrows.meta} % need this for arrow tips
\pgfplotsset{width=10cm,compat=1.14}
\usetikzlibrary{datavisualization}
\usepgfplotslibrary{statistics} 
\usepackage{Times}

\geometry{height=8in, width=5in}

\usepackage[utf8]{inputenc}

\setlength{\epigraphwidth}{.8\textwidth} \setlength{\epigraphrule}{0pt}

\urlstyle{same}

\lefthyphenmin=2
\righthyphenmin=3

\brokenpenalty=10000 % No broken words across columns/pages
\widowpenalty=10000 % No widows at bottom of page
\clubpenalty=10000 % No orphans at top of page

\begin{document}
\maketitle

\section*{Introduction} 

\epigraph{Twenty-first century constitutionalism is redefining the long tradition of expert constitution making and bringing it into the sphere of democratic participation}{\emph{Vivien Hart}}
\epigraph{``a successful process is transformational; it converts the spoilers''}{\emph{quoted by Jennifer Widner}}

The (re)writing of constitutions is, today, a norm in post-conflict peace building operations (Landau and Lerner 2019, 4; Brandt 2005; Benomar 2003). A reason for its popularity rests on its promulgation by the international community, which has required internationally brokered agreements to include reforms to the constitution. On the demand side, post-conflict states have also become more pro-active about engaging in constitution writing due to the ``democratic credentials'' which come with the process, and the international recognition or aid that subsequently follow from such credentials (Hart 2003, 2). Within post-conflict countries, political contenders and citizens have also called for constitutional reforms to legally enshrine their rights and to constrain a state that has committed large-scale violence against a population (Böckenförde, Hedling and Wahiu, 2011).

More than just the terms of the constitution, which can have strong legal and political power on all bounded actors due to the supremacy of the constitution as the higher law, the process of constitution-writing has also gained attention due to the changes that have come out of more recent negotiations over the constitution (Landau and Lerner 2019; Widner 2008; Samuels 2006; Hart 2003). One of these differences is the democratization of the process, and specifically, the move away from an elite-centric bargaining process toward the involvement of the wider public (Saunders 2014; Brandt, Cottrell, Ghai and Regan 2011).

The chief theoretical premise motivating this shift in practice is twofold. The first is to prevent negotiating parties -- primarily the conflict actors -- from taking the opportunity to divide the spoils of office. The second is to ensure a fair representation so that all parties bound by the constitution, which include the wider public, would regard the document as legitimate, and hence support the terms within it (IDEA 2014; Saunders 2014; Brandt et al 2011).

This chapter pursues this line of argument to investigate the relationship between the democratization of constitution-writing and the quality of elections, that is, the extent of (non)violence during an electoral cycle, in a post-conflict environment. Broadly speaking, if democratization encourages all members of a country to acknowledge the legitimacy of the constitution, and hence the terms -- including electoral ones -- enshrined within it, then they should be inclined to comply with the non-violent procedures of an election, even if they fall on the side of the non-elected camp. Yet, empirical support for the relationship between popular participation and post-conflict peace or democracy is mixed (Landau 2013; Widner 2008).

To refine the democratization argument, the argument here highlights the importance of reconciling between broad-based participation and traditional hierarchies to prevent the latter from eroding the legitimacy of the constitution. Leaders who might defend the traditional sources of political authority include warlords, chiefs and clan leaders. When all members of a society are able to form groups to advocate for themselves, the social relevance of these traditional leaders can be undermined. Fearing a loss of authority, these leaders may turn toward disrupting the post-conflict peace, for instance, by intimidating voters to prevent them from supporting or taking part in the post-conflict system. Alternatively, as more individuals or groups become empowered to join politics, leaders who had never faced stiff competition, if at all, at the local level may experience a threat to their political power. To discourage political participation, leaders may likewise turn to force either to instill fear or to demonstrate their superior ability to provide security (Daly 2019; Mukhopadhyay 2014).
% that broad-based participation may not in fact be favored by every segment of the society (Samuels 2006, 670-1; on some other pitfalls of the participatory process, see Brandt et al 2011, 87-90).

Thus, while the democratization of the constitution-writing process may be encouraged, special attention has to be given to the inclusion of the traditional leaders. Although democratization by itself implies that all groups may freely seek representation, traditional leaders may be sidelined, intentionally or not, if they are seen as a threat to the new vision of a centralized goverment or a bulwark against the ideal break from corruption and nepotism. As Roger McGinty's (2010, 590) account of Afghanistan's experience suggests, the co-optation of warlords contradicted the ``Weberian merits-based criteria'' that was the preferred by external interveners. Hence, international pressure remained on the Karzai administration to ``clean up'' the government -- until US security interests in the country changed.

The point is that unless traditional leaders are included in the constitution-making process, democratization efforts to enhance popular support for a constitution may end up undermining its legal authority within a country. Refusing to accept the terms of the constitution, these groups may challenge the institutions agreed upon in the constitution-making process to express their unwillingness to be bound by the constitution. If so, a democratic constitution-making process can lead to low quality elections due to violence down the road.

This chapter contributes to the existing literature by investigating the mediated effect between a democratic constitution-writing process and the quality of elections. Specifically, it suggests that the inclusion of traditional leaders is necessary to prevent them from becoming spoilers during elections. The inclusion of traditional leaders may bring about a ``hybridised legitimacy'' or a ``hybridised political order'' (Boege 2014; Landau 2013; Lund 2006) that can be unacceptable to Western observers because it represents a step away from the ideal, rational-legal legitimation. Yet, the downsides notwithstanding, support by traditional leaders for a constitution can not only reduce violence but encourage their constituents to also adopt the same position, an occurrence observed in some parts of Uganda (Moehler 2006), which may facilitate peace and democracy in the longer term. % OECD 2010

The following section continues with a discussion on democratic constitution-making. The third section  expands on the argument and describes the research design. The fifth section concludes with some next steps in this research effort.

\section*{Democratic Constitution-Making}

Constitutions and constitution-writing, by themselves, are not necessarily democratic. Non-democracies may write constitutions to limit political competition and marginalize pockets of a population. Anti-democratic leaders in democratic regimes may pursue constitutional changes to expand executive limits and consolidate power (Halmai 2019). However, when constitution-writing is brought to post-conflict states, the perspective held by peacebuilding practitioners is that of constitution-writing as a sign of a state's ``democratic mandate'', in addition to its value as an ``exit point'' for external interveners (IDEA 2011, 11). \footnote{The definition of constitution-making remains debated. This chapter sees constitution-making more so as a process of revision, which Richard Albert (2019, 117) defines as ``a formal constitutional change that departs from our understanding of what the constitution means and indeed allows by its spirit and design.'' While constitution-making may include amendments, which are changes that remain within the ``established framework'', it is likely the major changes that have a greater effect on the power relations in post-conflict states. The effect of minimal or no change can be an interesting research puzzle but it falls outside of the scope of this chapter.}

As constitution-making evolves to meet challenges in war-torn states (e.g. to prevent constitutions from becoming a platform for institutionalizing the spoils of war), the emphasis on a democratic process furthermore increases, so much so that democratization is now a feature of modern day constitutional assistance (Landau 2013, 933-4; Williams 2013, 13). In what may be termed as ``democratic constitution making'', the process comprises now, according to Vivienne Hart (2003, 4),

\begin{quote}
\small
a moral claim to participation, according to the norms of democracy. A claim of necessity for participation is based on the belief that without the general sense of ``ownership'' that comes from sharing authorship, today’s public will not understand, respect, support, and live within the constraints of constitutional government. Whether there is also a legal right to participate, for whom, and what all of this means in practical terms, are also key issues for modern constitutionalism, whose reputation and effectiveness
depend upon democracy in its process as well as its outcome.
\end{quote}

Constitution-writing, in short, is as much a state-building instrument as a democratic tool that is infused with the ``norms of democratic procedure, transparency, and accountability'' (Wallis 2019; Ludsin 2011; Hart 2003, 4). \footnote{This chapter assesses only the effect of constitution-making on peace and democracy in post-conflict environments. For a critique of whether this widespread policy practice is fundamentally compatible with peacebuilding, see Ludsin (2011).}

Many arguments can be made to merit the democratization of the constitution-writing process. For instance, a fairer representation of the diversity in a society can encourage groups to use the process to resolve differences and to establish a national unity that cuts across various cleavages. Empowerment of different segments of a society, especially those that would otherwise be under-represented or not at all, can improve all groups' understanding of the constitution while increasing their capacity to engage in politics and defend their rights subsequently. Finally, public participation can enhance a population's sense of ownership toward the constitution, and hence be more willing ``to defend it against sabotage''. (Saunders 2014; Landau 2013, 933-4; Brandt et al 2011, 86-7).

However, the same process have also suffered from a variety of problems in practice (Landau 2013; Brandt et al 2011, 87-90). This chapter focuses on one, which suggests that mass participation in the deliberation of the constitution can pose a threat to established power structures, and hence, lead the relevant leaders to undermine the constitution or prevent its enforcement (Samuels 2006, 670-1).

\section*{Bringing Traditional Leaders into the Constitution-Making Process}

This chapter posits that, in post-conflict settings, the effect of democratic constitution-making on electoral quality is mediated by the inclusion or exclusion of traditional leaders. Although the goal of democratic participation is to ensure that a diversity of opinion and needs is heard, international interveners and conflict actors can be motivated to exclude traditional leaders from the political process. A reason for their exclusion include the fear that these leaders may use their military capacity to demand greater power, thereby resulting in a constitution that either deepens the unequal power distribution across the society or that weakens the central state.

Separately, these leaders may also be excluded if they are regarded as a bulwark against the establishment of a rational-legal bureaucracy. One concern is that a bureaucracy that is not staffed through merit may result in an inefficient government. Poor service delivery and unclear promotion criteria can undermine public confidence, and hence, trust in the new and supposed democratic system (Dagher 2018). Alternative means of staffing the bureaucracy or appointing government officials may also perpetuate problems of corruption, nepotism or economic inequality, which are likely drivers of conflict (Walters 2015; Lindberg and Orjuela 2014; Hegre 2012; North, Wallis, Webb and Weingast 2010).

However, when traditional leaders are excluded, they can become fearful of becoming socially or politically marginalized as the rest of the population is becoming politically empowered. Social marginalization can occur if these leaders' constituents become adept at forming associations and making demands to the central government. When this happens, constituents can choose to bypass these leaders, thus rendering them irrelevant for the provision of the community's needs.

Political marginalization can also take place at the level of gubernatorial elections. As more citizens become politically aware, more individuals may be motivated to form or join parties to contest in elections. Thus, leaders competing at the subnational level may be forced to face an unprecedented level of competition, which increases their probability of being defeated. The fear of replacement is likely worse where leaders have never faced the threat of replacement (Mukhopadhyay 2014).

As exclusion leads to uncertainty, which leads to fear, leaders may eventually be driven to challenge the legitimacy of the constitution by refusing to comply with the terms therein contained. At the electoral level, such challenge may mean the use of force to secure one's position or to dispute dissatisfactory outcomes considering that leaders might see the electoral rules stipulated within the constitution as non-legally binding.

Given the consequences of exclusion, this chapter argues that the inclusion of traditional leaders may be necessary to prevent them from using extensive violence to undermine the quality of elections. As Andrew Watkins (n.d.) notes, the co-optation of warlords in Afghanistan, Mogadishu and Somalia had allowed these central governments to exercise a greater degree of control over the warlords than was otherwise possible. To note, the inclusion of traditional leaders inconstitution-making does not necessarily equate to the inclusion of such terms that reflect the interests of these leaders. Whether such terms are included should be outcome of the negotiation, not a promise to be made upfront. Consultation with these leaders can be pursued on a different track to ensure that their interests are heard and that they are kept in the loop, so that they are politically involved in the process. The logic is not different than the underlying premise for a democratic process. Rather, the cautionary tale is to guard against their exclusion and the consequences that might come with it.
% include a paragraph about legitimacy -- legal and consent, or challenges to legitimacy -- ~ legal and ~ consent?

Nevertheless, there are risks to including the traditional leaders of a society. In Afghanistan, rather than bringing warlords into the fold of the central goverment, strong warlords maintained the capacity to challenge the central state due its relative weakness and need to rely on external support (McGinty 2010, 589). Conferring an official status upon ```wrong' local actors'' (Saunders 2004, 57) had also led to questions of justice as some warlords were believed ``to be involved in the deaths of hundreds of Taliban prisoners in late 2001'' (McGinty 2010, 589). As such, the inclusion of traditional leaders is, at best, a necessary but not sufficient condition for engendering non-violent behavior.

\section*{Research Design} % insert figure on elections - elevated to account for violence and non-violence, and mechanisms

In this argument, the independent variable is the democratization of the constitution-making process. There can be different aspects to the democratic process (Interpeace 2015), but the primary feature of concern in this chapter is the extent of participation. Accordingly, a more democratic process is one that includes a greater array of groups.

To measure this variable, this chapter looks at the following indicators gathered from the Comparative Constitutions Project (CCP): (1) number of groups protected from discrimination; and (2) restriction of rights of any groups (Elkins, Ginsburg and Melton 2014). Figure ~\ref{fig1} shows how this variable is aggregated. Addition is used at the indicator level because the two indicators are somewhat substitutable. Although the groups included in each question differ, they reflect the extent to which groups are included or excluded from the final terms of the constitution.

\begin{figure} [h!]

\caption{Extent of Democracy of the Constitution-Making Process}
\label{fig1}% 
\begin{center} 
\small
\includegraphics[width=\linewidth]{IV.jpg}

{\footnotesize Double dashed line: Ontological relationship}

{\footnotesize Arrow: Indicator of}

{\footnotesize + : Addition}
 
\end{center}
\end{figure}

There are two limitations to using the indicators from the CCP. First, they are not consistent with the unit of analysis in the chapter. Rather than constitutional bargains, the CCP records the terms of the constitution, which are outcomes of the negotiations. Second, the choice of indicators assumes that the number of groups protected corresponds with the diversity of groups that were represented in the constitution-making process. While protection for a group might not be included in a constitution had the group not been involved, the correspondence is not a given. International actors can demand that conflict actors include a particular clause (e.g. women's rights, which is increasingly a norm in peace agreements), and conflict actors can include stipulations chiefly based on trends that appear to increase one's democratic credentials or that fulfil the conditions tied to political aid.

Hence, the use of the independent variable in this chapter is primarily to explore the descriptive relationship between democratization and electoral quality. In the case study analysis, which is the primary research method used in this chapter, the extent of participation will be properly accounted for after considering the diversity within the country and the groups that are included in or excluded from the process. The mediating factor, the inclusion or exclusion of traditional leaders, will also be measured only during the case study analysis due to the lack of data and the highly context-specific nature of the variable.

The dependent variable, electoral quality, is defined as the extent of (non)violence of an electoral cycle. Extent, more specifically, refers to the balance between violence (i.e. the threat or actual use of force) and non-violence, and the fatality due to such violence. Measures on violence and non-violence will be put together from the Varieties of Democracy (V-Dem) project's elections index, and data on the number of deaths throughout the electoral cycle will be collected from the Electoral Contention and Violence (ECAV) dataset.

Figure ~\ref{fig2} summarizes the variable. This variable is aggregated across all levels via multiplication because the indicators and dimensions are non-substitutable. If an election is non-violent, the quality will be high; conversely, if an election is shrouded in violence, the quality will be low. Within the subset of violent elections, those with higher fatality rates would furthermore register a lower electoral quality. To ensure that the indicators and the quality vary in the same direction, the scaling of the variable will follow a fuzzy logic where an election with the highest quality will receive a 1.0 membership while the lowest will receive a 0.0. The actual scaling will depend on the distribution of the cases and case knowledge.

\begin{figure} [h!]

\caption{Electoral Quality}
\label{fig2}% 
\begin{center} 
\small
\includegraphics[width=\linewidth]{DV.jpg}

{\footnotesize Double dashed line: Ontological relationship}

{\footnotesize Arrow: Indicator of}

{\footnotesize * : Multiplication}
 
\end{center}
\end{figure}

\section*{Conclusion}

This chapter explores the relationship between the extent of democratization of a constitution-making process and the quality of an election in post-conflict states. Like some in the literature, this study argues that there is not a clear-cut relationship between mass participation in constitution-making and peace or democracy. Rather, whether an election is non-violent and can be considered democratic is mediated by whether traditional leaders are involved in the constitution-making process.

Some next steps would include better specifying the research and case selection methods. More thought should also go into measuring the independent variable to potentially overcome the existing limitations. Likewise, more work needs to go into reading cases to develop a better sense of what might constitute a high or low degree of participation both within a society, and the inclusion of the traditional leaders.

\end{document}