\documentclass [11pt]{article}

\title{Explaining Post-Conflict Democratic Progress: Logic of Democratic State Formation} 
\author{Kimberly Peh}
\date{\today}

\usepackage{hanging,verbatim,geometry,rotating,graphics,epigraph,afterpage,url,pdfpages,pifont}
\usepackage{etex,tikz,xcolor,verbatim,geometry,afterpage,float}
\usepackage{pgfplots}
\usepgflibrary{arrows.meta} % need this for arrow tips
\pgfplotsset{width=10cm,compat=1.14}
\usetikzlibrary{datavisualization}
\usepgfplotslibrary{statistics} 
\usepackage{Times}

\geometry{height=8in, width=5in}

\usepackage[utf8]{inputenc}

\setlength{\epigraphwidth}{.8\textwidth} \setlength{\epigraphrule}{0pt}

\urlstyle{same}

\lefthyphenmin=2
\righthyphenmin=3

\brokenpenalty=10000 % No broken words across columns/pages
\widowpenalty=10000 % No widows at bottom of page
\clubpenalty=10000 % No orphans at top of page

\begin{document}
\maketitle

\section*{Introduction} 

How do post-conflict states achieve democratic progress? Since the end of the Cold War, practitioners and scholars of peacebuilding have touted both peace and democracy as the twin goals of post-conflict peacebuilding (Manning and Smith 2016; Fortna and Howard 2008; Hippler 2008; Reilly 2008; Carothers 2007; Boutros-Ghali 1992). However, while a consensus may be reached on the goals, a debate persists on the `best' way by which these goals may be achieved (Walters 2015; Jarstad and Sisk 2008; Carothers 2007; Paris 2004).

Two main approaches characterize the core of the debate. The first may be called the sequentialist paradigm. Leading this school of thought are scholars who, after reflecting on the effects of external interventions in inter- and intra-state wars, argue that early democratization efforts are counterproductive to the pursuit of either peace or democracy (e.g. Reilly 2017; Brancati and Snyder 2012; Grimm 2008; Mansfield and Snyder 2005; Paris 2004; Zakaria 1997). These ``early democratization efforts'' refer most often to the conduct of elections, which these scholars emphasize is a fundamentally divisive form of politics that may not only entrench social and political cleavages but also ease the mobilization of resources for the renewal of war (Letsa 2017; Paris 2004). As such, scholars advocate for the ``Institutionalization before Liberalization (IBL)'' (Paris 2004) or the state-building approach, which prioritzes the building of political, economic and secuirty institutions before the implementation of democratic tools. Here, the assumption is that a strong state is necessary to mitigate the conflict-inducing effects of democratization (Grimm and Merkel 2008).

The second paradigm may be termed the gradualist approach (Carothers 2007). This framework espouses the concurrent implementation of both state- and democracy-building tools. According to Thomas Carothers (2007), the sequentialist framework rests on shaky theoretical foundations and assumptions which do not justify the sequentialists' skepticism over the value of promoting democracy early on in post-conflict states. Others show that democratic behaviors can be habituated (e.g. Lindberg 2003), and hence, the early conduct of such democratic institutions as elections can lay the ground for democratization in the future. The gradualist approach is also that which is most commonly promoted and practiced by the United Nations (UN) and the United States. Across the range from conservatives to liberals and from doves to hawks, US presidents have used or endorsed the merits of democracy as a reason for intervention or a means of conflict resolution (Hippler 2008). Asserting its neutral stand, the UN cites the seemingly peaceful nature of democracies -- most often supported by the Democratic Peace theorists (e.g. Doyle 2005; Russett et al. 1995; Maoz and Russett 1993) -- as a justification for promoting democracy in post-conflict settings (Boutros-Ghali 1992). Due to the enthusiastic promotion of democracy, virtually all post-conflict countries today hold elections as a way to signify the end of a conflict and the time between the signing of an agreement and the conduct of an election reduced by approximately half (from 5.6 years to 2.7 years) after the end of the Cold War (von Borzyskowski 2019; Brancati and Snyder 2012).

Evidently, the prevailing debate casts differences between the approaches as irreconcilable. I argue that the seemingly antithetical nature of these approaches stems from the gradualists' narrow view of what constitutes democratic tools and the sequentialists' shortfall in regard to specifying the quality of institutions and the mechanisms they expect would link institutions to conflict prevention. Putting together the ideas from both frameworks, I advance the logic of democratic state formation, which posits that post-conflict states can attain democratic progress -- shifts towards meaningful elections and citizen participation -- when institutions that constitute a state take on a democratic character. By ``democratic character'', I mean the broad-based nature of politics and the constraints imposed on governments. Specifically, it refers to the availability of platforms (e.g. protests and constitutional courts) through which citizens may legally use to contest a government and evidence of such effort by a government to justify its right to rule in exchange for citizens' support. Notably, the relationship between having a democratic character and democratic progress is not tautological. Many countries in North America and Europe today provide citizens with forums to contest governments and require that parties renew their mandate at least during each electoral cycle; yet, even these countries are not immune to the problem of autocratization or backsliding.
% Theoretical foundation -- Weber, Tilly, Hui, Wood, Huang, Wantchekon, O'Donnell (brown, blue and green zones -- only look at the legal dimension)

This argument is not entirely new. Scholars have used the state formation lens to study post-conflict peacebuilding (Karim 2020; Sosnowski 2019; Shinoda 2018) and the institutions that I will use to define a state are common explanatory variables (e.g. demobilization and the establishment of courts) in the post-conflict democratization literature. Still, the democratic formation thesis adds value to the ongoing debate. It synergizes a debate that highlights differences over similarities; logically brings together the presently disparate findings; and provides a clear causal link between institution-building and democratization in post-conflict settings. These contributions are crucial because the state-building paradigm so far largely asserts the necessity of different institutions and falls short of providing an empirically tested causal chain. While trying to fulfill these theoretical goals, this paper also corrects for some methodological issues in the literature by combining Qualitative Comparative Analysis (QCA) with case study analysis to overcome existing limitations in explaining post-conflict democratic progress. This chapter's focus on portable mechanisms and theoretical logics can also shed light on policies. If the argument holds true, it lends support to those who call for external interveners to have a more comprehensive plan; provides interveners with a framework to evaluate such a plan; and joins some scholars in urging interveners to focus also on bottom-up paths to democratization (Schmidt 2020; Rao 2013; Wood 2001; Hippler 2008).

The following section continues the discussion by outlining the problems in the existing literature on post-conflict democratization. The subsequent section then elaborates on the argument and the theoretical foundations that inform the basis of the logic. The fourth section will describe the research method, which includes a discussion on QCA, the case selection logic and the conceptualization of the independent and dependent variables (i.e. democratic state formation and democratic progress). The fifth section concludes with some next steps in this research effort.

% a section on why study post-conflict democratic progress?

\section*{The Sequentialist Approach to Post-Conflict Democratization}

An interesting characteristic of the literature on post-conflict peacebuilding is that the goals of democracy and peace are often studied separately. Scholars who are most concerned with the duration of post-conflict peace usually focus on the risk or likelihood of conflict recurrence (e.g. Letsa 2017; Walter 2015; 2004; Brancati and Snyder 2012), while others narrow in on the prospect of democratic transitions in post-conflict states (e.g. Lyons 2016; Joshi 2010; Gurses and Mason 2008; Wantchekon 2004; Wood 2001). When both are assessed in conjunction, the questions asked regard the extent to which democratization efforts hinder peace, and whether the means of conflict management hamper the possibility of future democratization. The answer, frequently and pessimistically, is that the pursuit of one tends to generate unintended repercussions on the other (see Jarstad and Sisk 2008, and specifically Höglund 2008). While this research angle, which emphasizes trade-offs, has provided valuable insights to policymakers and has advanced post-conflict research, it has also as a consequence, a blind spot where potential synergies are concerned.

Notwithstanding its strengths, this literature also suffers from several theoretical and methological limitations, of which, three will be discussed in this section. The first and second relate to the arguments put forth by proponents of the state- or the institution-building paradigm. A key assumption underlying this argument is that a strong state, composed of strong institutions, is a necessary condition for democracy (Brancati and Snyder 2012; Grimm and Merkel 2008; Hippler 2008; Paris 2004; Kumar 1998). Some scholars cite, for example, Larry Diamond (2006: 94), who writes that `Before a country can have a democratic state, it must first have a state.'' Others quote Juan Linz and Alfred Stepan (1996, 17), who famously state that `without a state, no modern democracy is possible.'' Still others refer to Samuel Huntington (1968: 4), who highlights the relationship between violence and the ``slow development of political institutions.'' In this research agenda, state-building is therefore the priority that is to precede democratization (hence the state-building approach is also known as the sequentialist approach), and it is, importantly, a necessary condition for preventing democratic efforts (specifically, elections) from undermining peace and democracy (Reilly 2017; de Zeeuw 2008). Yet, research thus far has not truly tested the necessity of the state as a precondition. Statistical analyses using hazard models and logit regression are the norm, and they test rather the extent to which institutions such as demobilization programs, power-sharing agreements, and rule of law, influence recurrence or democratic outcomes than the degree of constraint these institutions impose on the possibility of attaining these goals (see Goertz and Mahoney 2012).

Dawn Brancati and Jack Snyder (2012, 828-9) suggest an alternative to the necessary condition perspective. According to them, several individually necessary conditions may be jointly sufficient in facilitating the success of post-conflict elections. They write,

\begin{quote}
\small
Early elections are less risky when one side has won a decisive military victory since the losing side lacks the ability to return to fighting if it fares poorly in the election. In the absence of a decisive victory, demobilization of one or both sides, or their integration into a new army, can mitigate the risk of early elections. Successful demobilization is a complicated and lengthy process \dots Strong bureaucratic institutions and generous financing are needed to facilitate demobilization. The development of robust administrative institutions, international peacekeeping, and economic development can also facilitate this process (Doyle and Sambanis 2006; Fortna 2008b). Of these factors, peacekeepers are more likely to be in place if elections are held soon after war ends, whereas the other factors are more likely to be favorable if elections are held later.
\end{quote}

Carrie Manning and Ian Smith (2016, 973) make clear the gap between such a theoretical view and the methodological practice when they state that ``[w]hile most statistical treatments assume that independent variables are largely independent of one another, we know that empirically the variables we identify are linked to one another in complex ways, and the impact of one factor is likely to be contingent on the presence of others.'' Indeed, despite recognizing the possibility of causal complexity (see Goertz 2017, Chapter 3), Dawn and Snyder (2012) proceeded with a logit analysis, which analyzes the net effect of each variable, \emph{ceteris paribus}. Overall, therefore, there is a theoretical and methodological gap in the literature wherein neither the necessity nor the jointly sufficient expectation of the explanatory conditions is tested.

A second and related issue is the lack of attention to the causal chain, specifically, the mechanisms that link the explanatory conditions to the outcome of democratization. While the sequentialist paradigm is born largely out of case studies, these cases were analyzed mostly to evaluate and criticize the democratization approach (e.g. Mansfield and Snyder 2005; Paris 2004). More recent studies took on a statistical turn and mechanisms that are proposed remain as such theoretical propositions. The vagueness regarding the causal chain led Thomas Carothers (2007), a champion of the gradualist approach, to publish a seething response to the sequentialists that point to the potential for strong states to take on an autocratic character. Carothers (2007) reasons that absent constraints on potential autocrats, strong state structures may be usurped by leaders to silence dissent and competition. This argument finds support in the state strength literature which implicitly perceives states with democratic regimes to be weak (Vu 2010; Skocpol 1985).

The third weakness of the state-building approach inheres in scholars' operationalization of ``post-conflict democratization''. The majority of existing scholarship on the topic uses data from the Polity IV project which, as Chapter 1 has shown, is an inappropriate measure for conflict research (an exception is Fortna and Huang 2012, who used Polity-X in their robustness check. On Polity-X, see Vreeland 2008). Furthermore, to calculate the degree of democratic change in post-conflict countries, scholars primarily take the difference between a country's pre- and post-conflict Polity scores. As Virginia Page Fortna and Reyko Huang (2012, 802) point out, a key problem with this operationalization strategy is that changes to democracy levels might have occurred prior to the onset of some explanatory variables, such as war outcomes, which violate a fundamental guideline of causal inference: temporal priority (Coppedge 2012).

This section's focus on the shortfalls of the sequentialist approach in no way suggests that the gradualist approach, which advocates for the simultaneous implementation of both state- and democracy-building programs, is perfect. Rather, its weaknesses are so well-documented in the literature that a rehashing of the points seems redundant (see for example, Reilly 2017; Flores and Nooruddin 2012; Jarstad and Sisk 2008; Paris 2004; Kumar 1998; Zakaria 1997). In fact, as mentioned, the sequentialist approach was introduced because democratization efforts, and especially the implementation of post-conflict elections, have undermined, if not jeopardized, the viability of achieving peace and democracy in post-conflict states. Hence, this chapter departs from the foundations of the state-building literature and seeks instead to fill the gaps that remain to be addressed.
% Stops at FPE, ~ progress <--> ST/LT effects (e.g. power sharing) - leaving it out because the problem is more pertinent to the second chapter

\section*{The Logic of Democratic State Formation}

% addresses here the problem of mechanisms (causal power and causal mechanism dimension -- the `thing' that is lacking in current discussion which opens the argument to attacks)


% state - Max Weber (1965, 1), a state is “a human community that (successfully) claims the \emph{monopoly} of the \emph{legitimate use} of \emph{physical force} within a given territory (emphasis added)”.
% ``legitimacy'' is more often left out implicitly or explicitly as an irrelevant dimension
% bringing legitimacy back in means bringing the democratic character of states back to the constitutive level of the definition of a state
% Huntington's point on institution (see Paris' critique in the IBL chapter)

% legitimate use has two dimensions - legal (O'Donnell) and perceived

% state formation -- *process* of becoming a state - Tilly, Hui (focus on the case study analysis; the process comes before the formation of a state, which is consistent with the aggregated finding that a democratic state is sufficient for democratization)



post-conflict states can attain democratic progress -- shifts towards meaningful elections and citizen participation -- when institutions that constitute a state take on a democratic character. By ``democratic character'', I mean the broad-based nature of politics and the constraints imposed on governments. Specifically, it refers to the availability of platforms (e.g. protests and constitutional courts) through which citizens may legally use to contest a government and evidence of such effort by a government to justify its right to rule in exchange for citizens' support.

% Research design - address the problem of necessity/jointly sufficient and measurement of democratization
% Why QCA is relevant - operationalization of the state brings together various explanatory variables that already exist in the literature
% Explain why democratic progress rather than democratization -- not transitions, and being honestly so, given the problems that plague the notion of `post-conflict', cite here Arias and Goldman, and the El Salvador and Guatemala cases -- cases which are touted by some (e.g. Wantchekon) as beacons of hope of democratization

% measurement
% Cederman, Hug and Krebs 2010 - democratic progress






























\end{document}
