\documentclass [11pt]{article}

\title{Explaining Post-Conflict Democratic Progress: Logic of Democratic State Formation} 
\author{Kimberly Peh}
\date{\today}

\usepackage{hanging,verbatim,geometry,rotating,graphics,epigraph,afterpage,url,pdfpages,pifont}
\usepackage{etex,tikz,xcolor,verbatim,geometry,afterpage,float}
\usepackage{rotating}
\usepackage{pgfplots}
\usepgflibrary{arrows.meta} % need this for arrow tips
\pgfplotsset{width=10cm,compat=1.14}
\usetikzlibrary{datavisualization}
\usepgfplotslibrary{statistics} 
\usepackage{Times}

\geometry{height=8in, width=5in}

\usepackage[utf8]{inputenc}

\setlength{\epigraphwidth}{.8\textwidth} \setlength{\epigraphrule}{0pt}

\urlstyle{same}

\lefthyphenmin=2
\righthyphenmin=3

\brokenpenalty=10000 % No broken words across columns/pages
\widowpenalty=10000 % No widows at bottom of page
\clubpenalty=10000 % No orphans at top of page

\begin{document}
\maketitle

\section*{Introduction} 

How do post-conflict states achieve democratic progress? Since the end of the Cold War, practitioners and scholars of peacebuilding have touted both peace and democracy as the twin goals of post-conflict peacebuilding (Manning and Smith 2016; Fortna and Howard 2008; Hippler 2008; Reilly 2008; Carothers 2007; Boutros-Ghali 1996). However, while a consensus may be reached on the goals, a debate persists on the `best' way by which these goals may be achieved (Walters 2015; Jarstad and Sisk 2008; Carothers 2007; Paris 2004).

Two main approaches characterize the core of the debate. The first may be called the sequentialist paradigm. Leading this school of thought are scholars who, after reflecting on the effects of external interventions in inter- and intra-state wars, argue that early democratization efforts are counterproductive to the pursuit of either peace or democracy (e.g. Reilly 2017; Brancati and Snyder 2012; Grimm 2008; Mansfield and Snyder 2005; Paris 2004; Zakaria 1997). These ``early democratization efforts'' refer most often to the conduct of elections, which these scholars emphasize is a fundamentally divisive form of politics that may not only entrench social and political cleavages but also ease the mobilization of resources for the renewal of war (Letsa 2017; Paris 2004). As such, scholars advocate for the ``Institutionalization before Liberalization (IBL)'' (Paris 2004) or the state-building approach, which prioritzes the building of political, economic and secuirty institutions before the implementation of democratic tools. Here, the assumption is that a strong state is necessary to mitigate the conflict-inducing effects of democratization (Grimm and Merkel 2008).

The second paradigm may be termed the gradualist approach (Carothers 2007). This framework espouses the concurrent implementation of both state- and democracy-building tools. According to Thomas Carothers (2007), the sequentialist framework rests on shaky theoretical foundations and assumptions which do not justify the sequentialists' skepticism over the value of promoting democracy early on in post-conflict states. Others show that democratic behaviors can be habituated (e.g. Lindberg 2003), and hence, the early conduct of such democratic institutions as elections can lay the ground for democratization in the future. The gradualist approach is also that which is most commonly promoted and practiced by the United Nations (UN) and the United States. Across the range from conservatives to liberals and from doves to hawks, US presidents have used or endorsed the merits of democracy as a reason for intervention or a means of conflict resolution (Hippler 2008). Asserting its neutral stand, the UN cites the seemingly peaceful nature of democracies -- most often supported by the Democratic Peace theorists (e.g. Doyle 2005; Russett et al. 1995; Maoz and Russett 1993) -- as a justification for promoting democracy in post-conflict settings (Boutros-Ghali 1992). Due to the enthusiastic promotion of democracy, virtually all post-conflict countries today hold elections as a way to signify the end of a conflict and the time between the signing of an agreement and the conduct of an election reduced by approximately half (from 5.6 years to 2.7 years) after the end of the Cold War (von Borzyskowski 2019; Brancati and Snyder 2012).

Evidently, the prevailing debate casts differences between the approaches as irreconcilable. I argue that the seemingly antithetical nature of these approaches stems from the gradualists' narrow view of what constitutes democratic tools and the sequentialists' shortfall in regard to specifying the quality of institutions and the mechanisms they expect would link institutions to conflict prevention. Putting together the ideas from both frameworks, I advance the logic of democratic state formation, which posits that post-conflict states can attain democratic progress -- shifts towards meaningful elections and citizen participation -- when institutions that constitute a state take on a democratic character. By ``democratic character'', I mean the broad-based nature of politics and the constraints imposed on governments. Specifically, it refers to the availability of platforms (e.g. protests and constitutional courts) through which citizens may legally use to contest a government and evidence of such effort by a government to justify its right to rule in exchange for citizens' support. Notably, the relationship between having a democratic character and democratic progress is not tautological. Many countries in North America and Europe today provide citizens with forums to contest governments and require that parties renew their mandate at least during each electoral cycle; yet, even these countries are not immune to the problem of autocratization or backsliding.
% Theoretical foundation -- Weber, Tilly, Hui, Wood, Huang, Wantchekon, O'Donnell (brown, blue and green zones -- only look at the legal dimension)

This argument is not entirely new. Scholars have used the state formation lens to study post-conflict peacebuilding (Karim 2020; Sosnowski 2019; Shinoda 2018) and the institutions that I will use to define a state are common explanatory variables (e.g. demobilization and legal fora) in the post-conflict democratization literature. Still, the democratic state formation thesis adds value to the ongoing debate. It synergizes a debate that highlights differences over similarities; logically brings together the presently disparate findings; and provides a clear causal link between institution-building and democratization in post-conflict settings. These contributions are crucial because the state-building paradigm largely tests these variables independently, asserts the necessity of different institutions, and falls short of providing an empirically tested causal chain.

While trying to fulfill these theoretical goals, this paper also corrects for some methodological issues in the literature by combining Qualitative Comparative Analysis (QCA) with case study analysis. This chapter's focus on portable mechanisms and theoretical logics can also shed light on policies. If the argument holds true, it lends support to those who call for external interveners to have a more comprehensive plan; provides interveners with a framework to evaluate such a plan; and joins some scholars in urging interveners to focus also on bottom-up paths to democratization (Schmidt 2020; Hayman 2014; Malik 2014; Rao 2013; Wood 2001; Lemay-Hébert 2009; Hippler 2008).
% see the NUS file on post-war peace-building and civil wars

The following section continues the discussion by outlining the problems in the existing literature on post-conflict democratization. The subsequent section then elaborates on the argument and the theoretical foundations that inform the basis of the logic. The fourth section will describe the research method, which includes a discussion on QCA, the case selection logic and the conceptualization of the independent and dependent variables (i.e. democratic state formation and democratic progress). The fifth section concludes with some next steps in this research effort.

% a section on why study post-conflict democratic progress?

\section*{The Sequentialist Approach to Post-Conflict Democratization}

An interesting characteristic of the literature on post-conflict peacebuilding is that the goals of democracy and peace are often studied separately. Scholars who are most concerned with the duration of post-conflict peace usually focus on the risk or likelihood of conflict recurrence (e.g. Letsa 2017; Walter 2015; 2004; Brancati and Snyder 2012), while others narrow in on the prospect of democratic transitions in post-conflict states (e.g. Lyons 2016; Joshi 2010; Gurses and Mason 2008; Wantchekon 2004; Wood 2001).

When both are assessed in conjunction, the questions asked regard the extent to which democratization efforts hinder peace, and whether the means of conflict management hamper the possibility of future democratization. The answer, frequently and pessimistically, is that the pursuit of one tends to generate unintended repercussions on the other (see Jarstad and Sisk 2008, and specifically Höglund 2008). While this research angle, which emphasizes trade-offs, has provided valuable insights to policymakers and has advanced post-conflict research, it has also as a consequence, a blind spot where potential synergies are concerned.

Notwithstanding its strengths, this literature also suffers from several theoretical and methological limitations, of which, three will be discussed in this section. The first and second relate to the arguments put forth by proponents of the state- or the institution-building paradigm. A key assumption underlying this argument is that a strong state, composed of strong institutions, is a necessary condition for democracy (Brancati and Snyder 2012; Grimm and Merkel 2008; Hippler 2008; Paris 2004; Kumar 1998; Boutros-Ghali 1996, 8). Some scholars cite, for example, Larry Diamond (2006: 94), who writes that `Before a country can have a democratic state, it must first have a state.'' Others quote Juan Linz and Alfred Stepan (1996, 17), who famously state that `without a state, no modern democracy is possible.'' Still others refer to Samuel Huntington (1968: 4), who highlights the relationship between violence and the ``slow development of political institutions.''

In this research agenda, state-building is therefore the priority that is to precede democratization (hence the state-building approach is also known as the sequentialist approach), and it is, importantly, a necessary condition for preventing democratic efforts (specifically, elections) from undermining peace and democracy (Reilly 2017; de Zeeuw 2008). Yet, research thus far has not truly tested the necessity of the state as a precondition. Statistical analyses using hazard models and logit regression are the norm, and they test rather the extent to which institutions such as demobilization programs, power-sharing agreements, and rule of law, influence recurrence or democratic outcomes than the degree of constraint these institutions impose on the possibility of attaining these goals (see Goertz and Mahoney 2012).

Dawn Brancati and Jack Snyder (2012, 828-9) suggest an alternative to the necessary condition perspective. According to them, several individually necessary conditions may be jointly sufficient in facilitating the success of post-conflict elections. They write,

\begin{quote}
\small
Early elections are less risky when one side has won a decisive military victory since the losing side lacks the ability to return to fighting if it fares poorly in the election. In the absence of a decisive victory, demobilization of one or both sides, or their integration into a new army, can mitigate the risk of early elections. Successful demobilization is a complicated and lengthy process \dots Strong bureaucratic institutions and generous financing are needed to facilitate demobilization. The development of robust administrative institutions, international peacekeeping, and economic development can also facilitate this process (Doyle and Sambanis 2006; Fortna 2008b). Of these factors, peacekeepers are more likely to be in place if elections are held soon after war ends, whereas the other factors are more likely to be favorable if elections are held later.
\end{quote}

Carrie Manning and Ian Smith (2016, 973) make clear the gap between such a theoretical view and the methodological practice when they state that ``[w]hile most statistical treatments assume that independent variables are largely independent of one another, we know that empirically the variables we identify are linked to one another in complex ways, and the impact of one factor is likely to be contingent on the presence of others.'' Indeed, despite recognizing the possibility of causal complexity (see Goertz 2017, Chapter 3), Dawn and Snyder (2012) proceeded with a logit analysis, which analyzes the net effect of each variable, \emph{ceteris paribus}. Overall, therefore, there is a theoretical and methodological gap in the literature wherein neither the necessity nor the jointly sufficient expectation of the explanatory conditions is tested.

A second and related issue is the lack of attention to the causal chain, specifically, the mechanisms that link the explanatory conditions to the outcome of democratization. While the sequentialist paradigm is born largely out of case studies, these cases were analyzed mostly to evaluate and criticize the democratization approach (e.g. Mansfield and Snyder 2005; Paris 2004). More recent studies took on a statistical turn and mechanisms that are proposed remain as such theoretical propositions. The vagueness regarding the causal chain led Thomas Carothers (2007), a champion of the gradualist approach, to publish a scathing response to the sequentialists that point to the potential for strong states to take on an autocratic character. Carothers (2007) reasons that absent constraints on potential autocrats, strong state structures may be usurped by leaders to silence dissent and competition. This argument finds support in the state strength literature which implicitly perceives states with democratic regimes to be weak (Vu 2010; Skocpol 1985).

The third weakness of the state-building approach inheres in scholars' operationalization of ``post-conflict democratization''. The majority of existing scholarship on the topic uses data from the Polity IV project which, as Chapter 1 has shown, is an inappropriate measure for conflict research (an exception is Fortna and Huang 2012, who used Polity-X in their robustness check. On Polity-X, see Vreeland 2008). Furthermore, to calculate the degree of democratic change in post-conflict countries, scholars primarily take the difference between a country's pre- and post-conflict Polity scores. As Virginia Page Fortna and Reyko Huang (2012, 802) point out, a key problem with this operationalization strategy is that changes to democracy levels might have occurred prior to the onset of some explanatory variables, such as war outcomes, which violate a fundamental guideline of causal inference: temporal priority (Coppedge 2012).

This section's focus on the shortfalls of the sequentialist approach in no way suggests that the gradualist approach, which advocates for the simultaneous implementation of both state- and democracy-building programs, is perfect. In fact, its weaknesses are well-documented in the literature (see for example, Reilly 2017; Flores and Nooruddin 2012; Jarstad and Sisk 2008; Paris 2004; Kumar 1998; Zakaria 1997), and as mentioned, the sequentialist approach was introduced precisely in reaction to such democratization programs as post-conflict elections and rebel-to-party transformations which have undermined the viability of peace and democracy in some post-conflict states (Reilly 2017; Lyons 2016; Ishiyama 2016; Brancati and Snyder 2012; Flores and Noorrudin 2012). Hence, this chapter departs from the foundations of the state-building literature and seeks instead to fill the gaps that remain to be addressed.
% Stops at FPE, ~ progress <--> ST/LT effects (e.g. power sharing) - leaving it out because the problem is more pertinent to the second chapter

\section*{Argument}

At the heart of Carothers' (2007) challenge to the sequentialist approach is the lack of a clear causal link between state-building and democracy. Currently, proponents of state-building mostly adopt an institutionalist approach, which privileges a ``state's ability to provide security'', its administrative capability and the ``ability of the state apparatus to affirm its authority'' (Lemay-Hébert 2009, 23). Unsurprisingly, the institutions that are tabled for consideration include demobilization programs, a police force, legal frameworks, a functioning bureaucracy, economic reforms and the like (Brancati and Snyder 2012; Paris 2004; Wantchekon 2004) -- institutions that concern ``the capabilities of the state to secure its grip on society (Lemay-Hébert 2009, 23).'' This focus on a state's authority and capacity finds its relevance in much of the state formation literature, which studies the two dimensions jointly under the notion of state strength (Kurtz 2013; Vu 2010; Mann 2008; Centeno 2002; Skocpol 1985; Tilly 1985).

The problem, as Carothers' (2007) critique suggests, is that such a `strong' state is as, if not more, likely to exhibit autocratic tendencies. A theory of democratic progress from the state-building perspective therefore requires a deeper discussion on how the development of a strong state can motivate a causal chain that leads to a democratic outcome. To do so, this section begins with assessing the conceptual basis that informs the sequentialist view of state-building. It contends that the relevant conceptualization of the `state' is missing a key dimension, legitimacy, which is the feature that bridges the link between state-building and democratic progress.

\subsection*{Conceptualizing the State}

A theoretical basis that aligns with the sequentialist view of state-building is the understanding of a state as an entity that monopolizes force (Tilly 1992; 1985, 171; Skocpol 1979; Hobbes 1651[1996]). This view essentially truncates Max Weber's (1965, 1) definition of a state as “a human community that (successfully) claims the monopoly of the legitimate use of physical force within a given territory” into that which exercises the monopoly of force (Vu 2010, 150). The concern as such lies with the coercive capacities of a state, which scholars argue makes possible societal acceptance or law and order -- features of the ``legitimate'' dimension in Weber's definition -- due to the consequent ability to provide security or public goods (Kurtz 2013; Tilly 1985, 172; Elias 1982; Hobbes 1651[1996]). Charles Tilly (1985, 171-2) states,

\begin{quote}
\small
Legitimacy is the probability that other authorities will act to confirm the decisions of a given authority. Other authorities, I would add, are much more likely to confirm the decisions of a challenged authority that controls substantial force; not only fear of retaliation, but also desire to maintain a stable environment recommend that general rule. \dots A tendency to monopolies (\emph{sic}) the means of violence makes a government's claim to provide protection, in either the comforting or the ominous sense of the word, more credible and more difficult to resist.
\end{quote}

Yet, a proper read of Weber's definition makes clear that this coercion-centric interpretation of a state is incomplete and it miscontrues the relationship between force and legitimacy. Weber (1965, 1) writes,

\begin{quote}
\small
In the past, the most varied institutions \dots have known the use of physical force as quite normal. \dots [A]t the present time, the right to use physical force is ascribed to other institutions or to individuals only to the extent to which the state permits it. The state is considered the sole source of the `right' to use violence. Hence, `politics' for us means striving to share power or striving to influence the distribution of power, either among states or among groups within a state.

Like the political institutions historically preceding it, the state is a relation of men dominating men, a relation supported by means of legitimate (i.e. considered to be legitimate) violence. If the state is to exist, the dominated must obey the authority claimed by the powers that be. When and why do men obey? Upon what inner justifications and upon what external means does this domination rest?
\end{quote}

From the excerpt, we see that Weber does not define a state by its ability to monopolize force; rather, what a state monopolizes is the `right' to use force, or what he succinctly calls `legitimate violence'. Accordingly, political contention with and over the identity of a state is about contesting who gets to exercise and share this right, and not about who possesses force per se. Without this dimension of rights, one's conceptualization is, thus, incomplete.

Relatedly, because a state is about having and monopolizing force, and the generation of legitimacy to use such force, legitimizing itself is an active part of a state's (re)constitution. Weber's paper, entitiled ``Politics as a Vocation'', is in fact an elaboration of the second quoted paragraph: the legitimation strategies of a state that allow states to justify their right to command force and hence, obedience. Therefore, properly understood, the relationship between force and legitimacy cannot be a causal one, which scholars seem to suggest.

The histories of Asia, Europe, and Latin America attest to the centrality of legitimacy in the process of state formation. Victoria Hui (2005, 48) finds that even the most aggressive Chinese rulers acceded to bargains by the peasantry to ensure that support for wars continued through an ongoing supply of food, taxes and recruits. In Europe, Charles Tilly (1992, 26) observes that it is ``inevitable'' for elites to extend rights to various groups to create buy-in and ``extract still more resources'' during periods of war preparation. In post-communist Europe, Latin America and East Asia, weak legitimacy is cited as a reason for the weakness of central states (North, Wallis, Webb and Weingast 2012; Ganev 2005; Centeno 2002).

Engagement of legitimation activities (e.g. rebel diplomacy and public goods provision) by clans, warlords, and rebels has also been described by scholars as attempts by non-state armed groups (NSAGs) to exhibit state-like qualities (Callimachi 2018; Arjona 2017; 2016; Huang 2016; Lara 2014; Kilcullen 2013). These activities sometimes influence civil war dynamics and outcomes by changing citizens' or international actors' willingness to recognize these groups' ``viability and authority'' (Jo 2015, 14). Recognition, in turn, can affect materiel support and hence, the balance of power or the extent to which defection or whistle-blowing occurs (Arjona 2016; Wood 2003; Kalyvas 2000). Thus, legitimacy is an important quality for actors that seek to be a state.

Based on this discussion, a complete conceptualization of a state thus comprises three dimensions: the fundamental existence of a physical force (this is a non-trivial dimension especially in a post-conflict setting that this section will explain later), its monopoly, and its legitimate use. Figure ~\ref{fig1} illustrates this conceptualization. It is only when an entity fulfills all three conditions (i.e. exists within the gray zone) that it may be called a state.

\begin{figure} [h!]

\caption{Conceptualization of the ``State''}
\label{fig1}% 
\begin{center} 
\small
\includegraphics[width=\linewidth]{Venn.jpg}
 
\end{center}
\end{figure}

\subsection*{Logic of Democratic State Formation} % non-violence






Not tautological - legitimation strategy and also, dependent on the balance among the three conditions, and whether the pursuit of legitimacy is temporary, as in China
And mass mobilization can lead to war/prevent wars from conclusion -- e.g. territorial wars where the vast majority does not see a stake/Letsa
social acceptance (can go both ways -- evident in the war preparation process, but it is part of the process that ultimately made the state), which is why fora for contestation are key
causal mechanism standpoint: Non-violence - to balance the coercive dimension; 




Definitions influence the choice of indicators


HOW - Mechanism
Joel Migdal writes, “the critical point is that … state and civil society are mutually reinforcing … It is the existence of widely held norms … or modes of social behavior … which reinforces the dominance of the state and allows it to rule without constant recourse to coercion … For the most part, it is the legal framework of the state that establishes the limits of autonomy for the associations and activities that make up civil society. If that framework is widely accepted, then the activities of the state and other social groups may be mutually empowering.”  Migdal’s point on the interaction between the state and the society, the limits that they impose on each other, and the mutually transforming effects that they thus trigger highlights that the state-centric approach presents at best a causal logic based on a static understanding of the state – in particular, the state that manifests the very moment and conditions that have shaped the state.

IMPORTANCE (non-violence and broad participation) --> democratization
Charles Tilly writes that “[s]truggle and bargaining with different classes in the subject population significantly shaped the states that emerged in Europe.” Not only is bargaining inevitable for elites who wish to extract still more resources from an otherwise unwilling population, each bargaining event such as “[p]opular rebellions … left marks on the state in the form of repressive politics, realignments of classes for or against the state, and explicit settlements specifying the rights of the affected parties.” 







% why each dimension is key in the post-conflict context

% Basic Framework and Operationalization
%\begin{sidewaysfigure}[ht]
%\caption{Caption in landscape to a figure in landscape.}
%\label{fig2}
%\begin{center} 
%\includegraphics[width=\textwidth]{figure.jpg}
  
%\end{center}
%\end{sidewaysfigure}

%{\footnotesize Fill in here.}



legitimate use has two dimensions - legal (O'Donnell) and perceived
% non-violence - the value for pursuing democratic tools, but that is left out of theoretical discussions when theorizing




post-conflict states can attain democratic progress -- shifts towards meaningful elections and citizen participation -- when institutions that constitute a state take on a democratic character. By ``democratic character'', I mean the broad-based nature of politics and the constraints imposed on governments. Specifically, it refers to the availability of platforms (e.g. protests and constitutional courts) through which citizens may legally use to contest a government and evidence of such effort by a government to justify its right to rule in exchange for citizens' support.

% Research design - address the problem of necessity/jointly sufficient and measurement of democratization
Why QCA is relevant - operationalization of the state brings together various explanatory variables that already exist in the literature
Explain why democratic progress rather than democratization -- not transitions, and being honestly so, given the problems that plague the notion of `post-conflict', cite here Arias and Goldman, and the El Salvador and Guatemala cases -- cases which are touted by some (e.g. Wantchekon) as beacons of hope of democratization

the three dimensions are necessary because -- Weber -- monopoly of the `sole source of right'


% measurement
Cederman, Hug and Krebs 2010 - democratic progress

FPE, as defined by Kumar, is consistent with the definition of the state proposed here because it marks the first officiation of mass participation in recognizing/endorsing the state

% conclusion - policy implications
Lemay-Hebert 2009 - implications of absence of local legitimacy on the success of external interventions (p. 37 onward)








% caveat against the romanticization of local peace (Richmond 2009); fallacy of a civil civil society


















\end{document}
